% Draw something looking like a files hierarchy.
% Not everything is possible yet, but it's maybe useful at least.
\tikzmath{
  \FileNameScale = 1.9;
  \FileSpacing = 3;
  \IconSize = 14;
}
\tikzset{file icon/.style 2 args={
  line width=1.5,
  rounded corners,
  line cap=round,
  every path/.style={fill=#1, draw=#2},
  },
  file icon +/.style={file icon={#1}{plus}},
  file icon -/.style={file icon={#1}{minus}},
  file icon 0/.style={file icon={#1}{unchanged}},
  file icon m/.style={file icon={#1}{modified}},
}

% Pad icons to given width.
\newcommand{\FileIconFrame}{\path[fill=none, draw=none] (0, 0)
  rectangle (\IconSize, 0);}

% Folder icon.
\NewDocumentCommand{\Folder}{ O{0} }{%
\tikz[file icon #1={Orange2}, baseline={(0,.35*\IconSize)}]{
  \FileIconFrame
  \path (0, 0) rectangle (0.4*\IconSize, .75*\IconSize);
  \path (0, 0) rectangle (0.85*\IconSize, .65*\IconSize);
  \begin{scope}[cm={1, 0, .25, .85, (0, 0)}] % Transformation matrix.
    \path (0, 0) rectangle (0.85*\IconSize, .65*\IconSize);
  \end{scope}
}}

% File icon.
\NewDocumentCommand{\File}{ O{0} }{%
\tikz[file icon #1={Light2}, baseline={(0,.45*\IconSize)}]{{
  \FileIconFrame
  \tikzmath{
    \W = .7*\IconSize;
    \H = .9*\IconSize;
    \o = .15*\IconSize; % horizontal offset
    \n = 4;   % number of lines
    \s = 2; % lines shortening
    \h = \H / (\n + 1); % lines spacing
  }
  \path (\o, 0) rectangle (\W+\o, \H);
  \foreach \i in {1,...,\n} {%
    \path (\o+\s, \i*\h) -- (\W+\o-\s, \i*\h);
  }
}}}

\makeatletter

% Label file or folder with a name.
% [keys]{icon-location}{filename}
\define@boolkey{NameFile}[]{file}[true]{}
\define@boolkey{NameFile}[]{folder}[true]{}
\define@cmdkey{NameFile}[]{mode}[0]{}
\presetkeys{NameFile}{mode}{}
\NewDocumentCommand{\NameFile}{ O{file} m m }{{
  \setkeys{NameFile}{#1}
  \node[anchor=base west, scale=\FileNameScale, \mode] (file-label)
    at ($(#2.south east) + (3, \iffile 5\fi\iffolder 3\fi)$)
    {\tt#3\iffolder/\fi};
  \edef\mode{\mode} % For some reason :\
  \ifdefstring{\mode}{0}{}{
    \node[anchor=base west, right=\SignOffset of file-label]
      {\ifdefstring{\mode}{m}{\Tilde}
      {\ifdefstring{\mode}{+}{\Plus}
      {\ifdefstring{\mode}{-}{\Minus}
      {[\mode]}}}};}
}}

% Step into folder hierarchy.
% [keys]{parent-icon-location}{name}{filename}
\define@boolkey{FirstChild}[]{file}[true]{}
\define@boolkey{FirstChild}[]{folder}[true]{}
\define@boolkey{FirstChild}[]{last}[true]{}
\define@cmdkey{FirstChild}[]{mod}[0]{} % for some reason recurses with NameFile@mode
\presetkeys{FirstChild}{mod}{}
\NewDocumentCommand{\FirstChild}{ O{file} m m m }{{
  \setkeys{FirstChild}{#1}
  \def\name{#3}
  \node[below=\FileSpacing of #2.south east, anchor=north west] (\name) {%
    \iffolder\Folder[\mod]\fi\iffile\File[\mod]\fi};
  \NameFile[\iffolder folder\fi\iffile file\fi, mode=\mod]{\name}{#4}
  \coordinate (\name-parent-slot) at ($(#2.south) + (-1, -2)$);
  \coordinate[left=2 of \name.base west] (\name-slot);
  \coordinate (\name-branch) at (\name-parent-slot |- \name-slot);
  \draw[file tree line, \iflast\mod\fi] (\name-parent-slot) -- (\name-branch);
  \draw[file tree line, \mod] (\name-branch) -- (\name-slot);
}}

% Append a new item to the current folder.
% [keys]{parent-icon-location}{name}{filename}
\define@boolkey{AppendSibling}[]{file}[true]{}
\define@boolkey{AppendSibling}[]{folder}[true]{}
\define@boolkey{AppendSibling}[]{last}[true]{}
\define@boolkey{AppendSibling}[]{connect}[true]{}
\define@cmdkey{AppendSibling}[]{mod}[0]{}
\presetkeys{AppendSibling}{mod}{}
\NewDocumentCommand{\AppendSibling}{ O{file} m m m }{{
  \setkeys{AppendSibling}{#1}
  \def\name{#3}
  \node[below=\FileSpacing of #2.south, anchor=north] (\name)
    {\iffolder\Folder[\mod]\fi\iffile\File[\mod]\fi};
  \NameFile[\iffile file\fi\iffolder folder\fi, mode=\mod]{\name}{#4}
  \ifconnect
    \coordinate (\name-sibling-slot) at (#2-branch);
    \coordinate[left=2 of \name.base west] (\name-slot);
    \coordinate (\name-branch) at (\name-sibling-slot |- \name-slot);
    \draw[file tree line, \iflast\mod\fi] (\name-sibling-slot) -- (\name-branch);
    \draw[file tree line, \mod] (\name-branch) -- (\name-slot);
  \fi
}}

\makeatother
