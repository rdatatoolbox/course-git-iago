\documentclass{standalone}

% The idea here, for now,
% is to construct a framework flexible enough
% to easily draw the various states of the files/tree/history,
% so it becomes easier to animate afterwards.
% This file will be typically (re-)generated from some other script.

\usepackage[english]{babel}
\usepackage[utf8]{inputenc}
\usepackage{mathptmx}
\usepackage[T1]{fontenc}

\usepackage{tikz}%
  \usetikzlibrary{math}
  \usetikzlibrary{positioning}
  \usetikzlibrary{calc}
\tikzset{
  x=1mm,
  y=1mm,
  inner sep=0,
}%

\usepackage{xcolor}

\definecolor{Blue1}{RGB}  {153,193,241}
\definecolor{Blue2}{RGB}  { 98,160,234}
\definecolor{Blue3}{RGB}  { 53,132,228}
\definecolor{Blue4}{RGB}  { 28,113,216}
\definecolor{Blue5}{RGB}  { 26, 95,180}
\definecolor{Green1}{RGB} {143,240,164}
\definecolor{Green2}{RGB} { 87,227,137}
\definecolor{Green3}{RGB} { 51,209,122}
\definecolor{Green4}{RGB} { 46,194,126}
\definecolor{Green5}{RGB} { 38,162,105}
\definecolor{Yellow1}{RGB}{249,240,107}
\definecolor{Yellow2}{RGB}{248,228, 92}
\definecolor{Yellow3}{RGB}{246,211, 45}
\definecolor{Yellow4}{RGB}{245,194, 17}
\definecolor{Yellow5}{RGB}{229,165, 10}
\definecolor{Orange1}{RGB}{255,190,111}
\definecolor{Orange2}{RGB}{255,163, 72}
\definecolor{Orange3}{RGB}{255,120,  0}
\definecolor{Orange4}{RGB}{230, 97,  0}
\definecolor{Orange5}{RGB}{198, 70,  0}
\definecolor{Red1}{RGB}   {246, 97, 81}
\definecolor{Red2}{RGB}   {237, 51, 59}
\definecolor{Red3}{RGB}   {224, 27, 36}
\definecolor{Red4}{RGB}   {192, 28, 40}
\definecolor{Red5}{RGB}   {165, 29, 45}
\definecolor{Purple1}{RGB}{220,138,221}
\definecolor{Purple2}{RGB}{192, 97,203}
\definecolor{Purple3}{RGB}{145, 65,172}
\definecolor{Purple4}{RGB}{129, 61,156}
\definecolor{Purple5}{RGB}{ 97, 53,131}
\definecolor{Brown1}{RGB} {205,171,143}
\definecolor{Brown2}{RGB} {181,131, 90}
\definecolor{Brown3}{RGB} {152,106, 68}
\definecolor{Brown4}{RGB} {134, 94, 60}
\definecolor{Brown5}{RGB} { 99, 69, 44}
\definecolor{Light1}{RGB} {255,255,255}
\definecolor{Light2}{RGB} {246,245,244}
\definecolor{Light3}{RGB} {222,221,218}
\definecolor{Light4}{RGB} {192,191,188}
\definecolor{Light5}{RGB} {154,153,150}
\definecolor{Dark1}{RGB}  {119,118,123}
\definecolor{Dark2}{RGB}  { 94, 92,100}
\definecolor{Dark3}{RGB}  { 61, 56, 70}
\definecolor{Dark4}{RGB}  { 36, 31, 49}
\definecolor{Dark5}{RGB}  {  0,  0,  0}


\tikzmath{
  \ScreenWidth = 400;
  \ScreenHeight = 300;
  \TitleBarHeight = 35;
  \TitleScale = 5.0;
  \SubTitleScale = 3.5;
  \PageNumScale = 2.0;
  \ProgressHeight = 1.5;
  \Progress = 1/3;
  \CanvasMargin = 5;
  \eps = 0.1; % Useful to avoid seemingly pixel-rounding errors.
}

\newcommand{\TitleText}{Title}
\newcommand{\SubTitleText}{Subtitle}
\newcommand{\PageNumText}{99}

\begin{document}


\begin{tikzpicture}

  % Set off global structure first.

  % Whole screen, setting the page size.
  \node[anchor=south west] (Screen) {\tikz{
    \path (0, 0) rectangle (\ScreenWidth, \ScreenHeight)}};

  % Title bar.
  \node[anchor=north] (TitleBar) at (Screen.north) {\tikz{
    \fill[Dark3] (0, 0) rectangle (\ScreenWidth, \TitleBarHeight);}};

  % Title.
  \node[Light2, anchor=base west, scale=\TitleScale,
        right=10 of TitleBar.west] (Title) {\bf \TitleText};

  % SubTitle.
  \node[Light2, anchor=base east, scale=\SubTitleScale,
        left=10 of TitleBar.east] (SubTitle) {\bf \SubTitleText};

  % Page number.
  \node[Dark4, inner sep=5, anchor=south east, scale=\PageNumScale]
    (PageNum) at (Screen.south east) {\bf \PageNumText};

  % Progress bar,
  % fix vertical borders white pixel lines with epsilon shifts.
  \coordinate (upper) at
    ($(TitleBar.south east) + (-\eps, \ProgressHeight)$);
  \coordinate[right=\eps of TitleBar.south west] (lower);
  \fill[Yellow1] (lower) rectangle (upper);
  \coordinate[left=(1-\Progress)*\ScreenWidth of upper] (upper);
  \fill[Orange1] (lower) rectangle (upper);

  % The "Canvas" refers to only the white area reserved for actual drawing,
  % minus a short margin.
  \coordinate (offset) at (\CanvasMargin, \CanvasMargin);
  \node[anchor=south west] at (offset) (Canvas) {\tikz{
    \path (offset) rectangle ($(TitleBar.south east) - (offset)$);}};

\end{tikzpicture}


\end{document}

