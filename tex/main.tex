\documentclass[multi=step]{standalone}

% The idea here, for now,
% is to construct a framework flexible enough
% to easily draw the various states of the files/tree/history,
% so it becomes easier to animate afterwards.
% This file will be typically (re-)generated from some other script.

\usepackage[english]{babel}
\usepackage[utf8]{inputenc}
\usepackage{amssymb}
\usepackage{droidsansmono}
\usepackage{mathptmx}
\usepackage[T1]{fontenc}
\newfontfamily\DroidSansMono{Droid Sans Mono}
\newcommand{\Code}[1]{{\DroidSansMono#1}}

\usepackage{xstring}
\usepackage{xkeyval}
\usepackage{etoolbox}
\usepackage{tikz}%
  \usetikzlibrary{math}
  \usetikzlibrary{positioning}
  \usetikzlibrary{calc}
  \usetikzlibrary{arrows.meta}
  \usetikzlibrary{intersections}
\tikzset{
  x=1mm,
  y=1mm,
  inner sep=0,
  % https://tex.stackexchange.com/a/447931/72679
  local to/.style={
      shift={(#1.center)},
      x={(#1.east)},
      y={(#1.north)},
  }
}
% Define a point wrt to existing node with relative coordinates.
% {node}{name}{location}
\newcommand{\IntensiveCoordinates}[3]{
  \begin{scope}[local to=#1]
    \coordinate (#2) at (#3);
  \end{scope}
}

\tikzmath{
  \eps = 0.1; % Useful to avoid seemingly pixel-rounding errors.
}

\usepackage{graphicx}
\usepackage{xsavebox}
\graphicspath{{./pictures}}

\foreach \bx/\filename in {
  Calzone/calzone.png,
  Capricciosa/capricciosa.png,
  ConsoleGit/console_git.png,
  Diavola/diavola.jpg,
  DuckFlames/duck_flames.png,
  DuckShy/duck_shy.png,
  Formation/Formation_Recherche_Reproductible.png,
  FRBCESAB/FRB-CESAB.jpg,
  GdREcoStat/GdR_EcoStat.jpg,
  GithubLogo/github_logo.png,
  GitIcon/git_icon.png,
  GitlabLogo/gitlab_logo.png,
  GitLogo/git_logo.png,
  Hazard/hazard.png,
  Heart/heart.jpg,
  ISEM/ISEM.png,
  Margherita/margherita.png,
  Marinara/marinara.jpg,
  Matrix/matrix.jpeg,
  MyMachine/my_machine.png,
  NowWhat/now_what.jpeg,
  OMG/omg.png,
  PullRequestButton/pull_request_button.jpg,
  Regina/regina.jpg,
  Relief/relief.jpg,
  RStudioExtension/rstudio_extension.png,
  Siciliana/siciliana.jpg,
  Skull/skull.pdf,
  Surprise/surprise.jpg,
  SyncForkButton/sync_fork_button.png,
  TheirMachine/their_machine.png,
  Think/think.jpg,
  VariousPizzas/pizzas_various.jpg,
  VSCodeExtension/vscode_extension.png,
}{\ifcsempty{bx}{}{%
  \xsavebox{\bx}{\includegraphics[width=1cm]{\filename}}
  \expandafter\xdef\csname Pic\bx\endcsname##1##2%
    {\resizebox{##1}{##2}{\xusebox{\bx}}}
}}

\usepackage{xcolor}

\definecolor{Blue1}{RGB}  {153,193,241}
\definecolor{Blue2}{RGB}  { 98,160,234}
\definecolor{Blue3}{RGB}  { 53,132,228}
\definecolor{Blue4}{RGB}  { 28,113,216}
\definecolor{Blue5}{RGB}  { 26, 95,180}
\definecolor{Green1}{RGB} {143,240,164}
\definecolor{Green2}{RGB} { 87,227,137}
\definecolor{Green3}{RGB} { 51,209,122}
\definecolor{Green4}{RGB} { 46,194,126}
\definecolor{Green5}{RGB} { 38,162,105}
\definecolor{Yellow1}{RGB}{249,240,107}
\definecolor{Yellow2}{RGB}{248,228, 92}
\definecolor{Yellow3}{RGB}{246,211, 45}
\definecolor{Yellow4}{RGB}{245,194, 17}
\definecolor{Yellow5}{RGB}{229,165, 10}
\definecolor{Orange1}{RGB}{255,190,111}
\definecolor{Orange2}{RGB}{255,163, 72}
\definecolor{Orange3}{RGB}{255,120,  0}
\definecolor{Orange4}{RGB}{230, 97,  0}
\definecolor{Orange5}{RGB}{198, 70,  0}
\definecolor{Red1}{RGB}   {246, 97, 81}
\definecolor{Red2}{RGB}   {237, 51, 59}
\definecolor{Red3}{RGB}   {224, 27, 36}
\definecolor{Red4}{RGB}   {192, 28, 40}
\definecolor{Red5}{RGB}   {165, 29, 45}
\definecolor{Purple1}{RGB}{220,138,221}
\definecolor{Purple2}{RGB}{192, 97,203}
\definecolor{Purple3}{RGB}{145, 65,172}
\definecolor{Purple4}{RGB}{129, 61,156}
\definecolor{Purple5}{RGB}{ 97, 53,131}
\definecolor{Brown1}{RGB} {205,171,143}
\definecolor{Brown2}{RGB} {181,131, 90}
\definecolor{Brown3}{RGB} {152,106, 68}
\definecolor{Brown4}{RGB} {134, 94, 60}
\definecolor{Brown5}{RGB} { 99, 69, 44}
\definecolor{Light1}{RGB} {255,255,255}
\definecolor{Light2}{RGB} {246,245,244}
\definecolor{Light3}{RGB} {222,221,218}
\definecolor{Light4}{RGB} {192,191,188}
\definecolor{Light5}{RGB} {154,153,150}
\definecolor{Dark1}{RGB}  {119,118,123}
\definecolor{Dark2}{RGB}  { 94, 92,100}
\definecolor{Dark3}{RGB}  { 61, 56, 70}
\definecolor{Dark4}{RGB}  { 36, 31, 49}
\definecolor{Dark5}{RGB}  {  0,  0,  0}

% Global frame structure.
% Invoke on every change to redraw.

\tikzmath{
  \ScreenWidth = 400;
  \ScreenHeight = 300;
  \TitleBarHeight = 35;
  \TitleScale = 5.0;
  \SubTitleScale = 3.5;
  \PageNumScale = 2.0;
  \ProgressHeight = 1.7;
  \CanvasMargin = 5;
  \LargeScale = 3.0;
  \NormalScale = 2.0;
}
\tikzset{
  progress made/.style={fill=Orange2},
  progress remaining/.style={fill=Yellow1},
}

% Uses \TitleText, \SubTitleText, \PageNumText.
% type=bare for non-regular steps with only a blank 'Screen'.
% type=title for transitions between slides.
% [type]{progress}{content}
\NewDocumentCommand{\Step}{ O{} m +m }{

\begin{step}%
\begin{tikzpicture}

% Whole screen, setting the page size.
\node[anchor=south west] (Screen) {\tikz{
  \path (0, 0) rectangle (\ScreenWidth, \ScreenHeight)}};

\IfSubStr{#1}{bare}{}{\IfSubStr{#1}{title}{

  \IntensiveCoordinates{Screen}{c}{0, .15}
  \node[scale=7, Dark3] at (c) {\TitleText};

  \IntensiveCoordinates{Screen}{lower}{-.7, -.005}
  \IntensiveCoordinates{Screen}{upper}{+.7, +.005}
  \path[progress remaining] (lower) rectangle (upper);
  \coordinate (mid) at ($(lower)!#2!(upper)$);
  \path[progress made] (lower) rectangle (mid|-upper);

}{


    % Tikz's 'z' system is not awesome, so many layers end up very specific :\
    \pgfdeclarelayer{highlight-behind}
    \pgfdeclarelayer{background}
    \pgfdeclarelayer{command-background}
    \pgfdeclarelayer{command-text}
    \pgfsetlayers{
      highlight-behind,
      background,
      main,
      command-background,
      command-text
    }

    % Title bar.
    \node[anchor=north] (TitleBar) at (Screen.north) {\tikz{
      \fill[Dark3] (0, 0) rectangle (\ScreenWidth, \TitleBarHeight);}};

    % Title.
    \node[Light2, anchor=base west, scale=\TitleScale,
          right=10 of TitleBar.west] (Title) {\bf \TitleText};

    % SubTitle.
    \node[Light2, anchor=base east, scale=\SubTitleScale,
          left=10 of TitleBar.east] (SubTitle) {\SubTitleText};

    % Page number.
    \node[Dark4, inner sep=5, anchor=south east, scale=\PageNumScale]
      (PageNum) at (Screen.south east) {\bf \PageNumText};

    % Progress bar,
    % fix vertical borders white pixel lines with epsilon shifts.
    \coordinate (upper) at
      ($(TitleBar.south east) + (-\eps, \ProgressHeight)$);
    \coordinate[right=\eps of TitleBar.south west] (lower);
    \path[progress remaining] (lower) rectangle (upper);
    \coordinate[left=(1-#2)*\ScreenWidth of upper] (upper);
    \path[progress made] (lower) rectangle (upper);

    % The "Canvas" refers to only the white area reserved for actual drawing,
    % minus a short margin.
    \coordinate (offset) at (\CanvasMargin, \CanvasMargin);
    \node[anchor=south west] at (offset) (Canvas) {\tikz{
      \path (offset) rectangle ($(TitleBar.south east) - (offset)$);}};

}}

#3

\end{tikzpicture}%
\end{step}

}

% Factorize bounding box highlighting procedures.
% [padding][opacity]{lower}{upper}
\NewDocumentCommand{\HighlightSquare}{ O{5} O{1} m m }{
  \begin{pgfonlayer}{highlight-behind}%
    \ifstrequal{#2}{1}{}{\begin{scope}[opacity=#2]}
    \coordinate (pad) at (#1, #1);
    \coordinate (lower) at ($(#3) - (pad)$);
    \coordinate (upper) at ($(#4) + (pad)$);
    \draw[Light5, line width=1, fill=Yellow1] (lower) rectangle (upper);
    \ifstrequal{#2}{1}{}{\end{scope}}
  \end{pgfonlayer}
}

% Hollow bounding box, useful for branches labels.
% [padding][opacity][color]{node}
\NewDocumentCommand{\HighlightSquareRing}{ O{1} O{1} O{Yellow1} m }{
    \coordinate (pad) at (#1, #1);
    \coordinate (lower) at ($(#4.south west) - (pad)$);
    \coordinate (upper) at ($(#4.north east) + (pad)$);
    \path[draw=#3, draw opacity=#2, line width=4] (lower) rectangle (upper);
}

% Highlight with background shade instead.
% [padding]{node}
\NewDocumentCommand{\HighlightShade}{ O{5} m }{
  \begin{pgfonlayer}{highlight-behind}%
    \coordinate (pad) at (#1, #1);
    \coordinate (lower) at ($(#2.south west) - (pad)$);
    \coordinate (upper) at ($(#2.north east) + (pad)$);
    \shade[inner color=Yellow1] (lower) rectangle (upper);
  \end{pgfonlayer}
}

% Automatic coordinates are interpreted as intensive to Canvas,
% unless they contain no comma (node name) or are a calculation with '$' or '|'.
% If they contain '=', they are interpreted as relative specifications
% like 'below=5 of node'
% {name}{location}
\newcommand{\AutomaticCoordinates}[2]{
  \IfSubStr{#2}{=}{
    \coordinate[#2] (#1);
  }{\ifboolexpr{test {\IfSubStr{#2}{$}} % Breaks syntax coloring -_-"
             or test {\IfSubStr{#2}{|}}
         or not test {\IfSubStr{#2}{,}}}
    {\coordinate (#1) at (#2);}
    {\IntensiveCoordinates{Canvas}{#1}{#2}}
  }
}


% Draw something looking like a files hierarchy.
% Not everything is possible yet, but it's maybe useful at least.
\tikzmath{
  \FileNameScale = 1.8;
  \FileSpacing = 16;
  \IntoFileTreeSpacing = 15;
  \IconSize = 14;
}
\tikzset{
  file icon/.style 2 args={
    line width=1.5,
    rounded corners,
    line cap=round,
    every path/.style={fill=#1, draw=#2},
    },
  file icon +/.style={file icon={#1}{plus}},
  file icon -/.style={file icon={#1}{minus}},
  file icon 0/.style={file icon={#1}{unchanged}},
  file icon m/.style={file icon={#1}{modified}},
  file tree line/.style={line width=1.5, line cap=round},
}

% Pad icons to given width.
\newcommand{\FileIconFrame}{\path[fill=none, draw=none] (0, 0)
  rectangle (\IconSize, 0);}

% Folder icon.
\NewDocumentCommand{\Folder}{ O{0} }{%
\tikz[file icon #1={Orange2}, baseline={(0,.35*\IconSize)}]{
  \FileIconFrame
  \path (0, 0) rectangle (0.4*\IconSize, .75*\IconSize);
  \path (0, 0) rectangle (0.85*\IconSize, .65*\IconSize);
  \begin{scope}[cm={1, 0, .25, .85, (0, 0)}] % Transformation matrix.
    \path (0, 0) rectangle (0.85*\IconSize, .65*\IconSize);
  \end{scope}
}}

% File icon.
\NewDocumentCommand{\File}{ O{0} }{%
\tikz[file icon #1={Light2}, baseline={(0,.45*\IconSize)}]{{
  \FileIconFrame
  \tikzmath{
    \W = .7*\IconSize;
    \H = .9*\IconSize;
    \o = .15*\IconSize; % horizontal offset
    \n = 4;   % number of lines
    \s = 2; % lines shortening
    \h = \H / (\n + 1); % lines spacing
  }
  \path (\o, 0) rectangle (\W+\o, \H);
  \foreach \i in {1,...,\n} {%
    \path (\o+\s, \i*\h) -- (\W+\o-\s, \i*\h);
  }
}}}

% Insert file/folder icon with a name and correct modification status.
% Type contains keywords like 'file|folder' and 'angle' checked with \IfSubStr.
% 'mod' is either '0+-m' TODO: c for 'conflict'.
% [type][mode]{location}{name}{filename}
\NewDocumentCommand{\FileLine}{ O{file} O{0} m m m }{{

  % Icon.
  \AutomaticCoordinates{#4}{#3}
  \node[anchor=north west,
        alias=\LatestFileTree-#4-icon] (#4-icon) at (#4)
    {\IfSubStr{#1}{file}{\File[#2]}{\Folder[#2]}};

  % Filename.
  \IfSubStr{#1}{file}{\coordinate (offset) at (3, 5);}
                     {\coordinate (offset) at (3, 3);}
  \node[anchor=base west, scale=\FileNameScale, #2,
        alias=\LatestFileTree-#4-filename] (#4-filename)
    at ($(#4-icon.south east) + (offset)$)
    {\Code{#5\IfSubStr{#1}{folder}{/}{}}};

  % Status sign.
  \ifdefstring{#2}{0}{}{
    \node[anchor=base west, right=\SignOffset of #4-filename]
      {\ifdefstring{#2}{m}{\Tilde}
      {\ifdefstring{#2}{+}{\Plus}
      {\ifdefstring{#2}{-}{\Minus}
      {[#2]}}}};} % (fallback to writing unexpected mode literally)

}}

% Chain files together, like commits in the repo.
% [anchor][name]{location}{type/mod/name/filename list}
\NewDocumentCommand{\FileTree}{ O{north west} O{files} m m }{

  % Update this global variable so that files icons, names, signs etc.
  % can be aliased with (filetreename-nodename) in addition to just (nodename).
  \def\LatestFileTree{#2}

  \AutomaticCoordinates{filetreeloc}{#3}
  \node[anchor=#1] (#2) at (filetreeloc) {\tikz[remember picture]{%
    \foreach \type/\mod/\name/\filename [count=\i, remember=\name as \lastname]
             in {#4}{\ifcsempty{filename}{}{%
      \ifnumcomp{\i}{=}{1}{
        % Keep track of current icon position.
        \coordinate (current) at (0, 0);
        \FileLine[\type][\mod]{current}{\name}{\filename}
      }{
        \coordinate[below=\FileSpacing of current] (current);
        % type with:
        %   'stepin' to enter into last folder.
        %   'connect' to append sibling into currently 'stepped in' folder.
        % TODO: 'step out' and continuing outer line within sub-subfolders.
        \IfSubStr{\type}{stepin}{
          \coordinate[right=\IntoFileTreeSpacing of current] (current);
        }{}
        \FileLine[\type][\mod]{current}{\name}{\filename}
        \IfSubStr{\type}{stepin}{%
          % First parenting line within the folder.
          \coordinate[below=2 of \lastname-icon.south] (s);
        }{\IfSubStr{\type}{connect}{%
          % Assumes an 'anchor' already exists.
          \coordinate (s) at (anchor);
        }{}}
        \ifboolexpr{ test {\IfSubStr{\type}{stepin}}
                  or test {\IfSubStr{\type}{connect}} }{
          % Draw the actual line, creating 'anchor' for next time.
          \coordinate[left=3 of \name-icon.west] (e);
          \coordinate (anchor) at (s|-e);
          % Style whole angle for '+' and '-' modes, otherwise only the tick.
          \ifboolexpr{ test {\ifdefstring{\mod}{+}}
                    or test {\ifdefstring{\mod}{-}}}{
            \draw[file tree line, \mod] (s) -- (anchor) -- (e);
          }{
            \draw[file tree line] (s) -- (anchor);
            \draw[file tree line, \mod] (anchor) -- (e);
          }
        }{}
      }
    }}
  }};

}

% Draw something looking like a diffed file.

\tikzmath{
  \SignRadius = 1.5;
  \SignBaseHeight = -1.5;
  \FileNameMargins = 2;
  \CodeScale = 1.4;
  \CodeMargins = 2;
  \SignOffset = 3;
}
\colorlet{plus}{Green5}
\colorlet{minus}{Red5}
\colorlet{unchanged}{Dark4}
\colorlet{modified}{Orange3}
\colorlet{conflict}{Purple3}
\tikzset{
  sign icon/.style={line width=2.0},
  +/.style={plus},
  -/.style={minus},
  0/.style={unchanged},
  m/.style={modified},
  c/.style={conflict},
}

% Plus icon.
\newsavebox{\PlusBox}
\savebox{\PlusBox}{\tikz[sign icon, every path/.style={+},
                         baseline={(0, \SignBaseHeight)}]{
  \draw (0, -\SignRadius) -- (0, \SignRadius);
  \draw (-\SignRadius, 0) -- (\SignRadius, 0);
}}
\newcommand{\Plus}{\usebox{\PlusBox}}

% Minus icon.
\newsavebox{\MinusBox}
\savebox{\MinusBox}{\tikz[sign icon, every path/.style={-},
                          baseline={(0, \SignBaseHeight)}]{
  \path (0, -\SignRadius) -- (0, \SignRadius);
  \draw (-\SignRadius, 0) -- (\SignRadius, 0);
}}
\newcommand{\Minus}{\usebox{\MinusBox}}

% Tilde icon.
\newsavebox{\TildeBox}
\savebox{\TildeBox}{\tikz[sign icon, baseline={(0, \SignBaseHeight)}]{
  \path (0, -\SignRadius) -- (0, \SignRadius);
  \tikzmath{
    \d = .2*\SignRadius;
    \a = 50;
    \l = 1.5*\SignRadius;
  }
  \coordinate (start) at (-\SignRadius, -\d);
  \coordinate (end) at  (\SignRadius, \d);
  \draw[m] (start)
        .. controls ($(start) + (\a:\l)$) and ($(end) + (180+\a:\l)$)
        .. (end);
}}
\newcommand{\Tilde}{\usebox{\TildeBox}}

% Lightning icon.
\newsavebox{\LightningBox}
\savebox{\LightningBox}{\tikz[sign icon, baseline={(0, \SignBaseHeight)}]{
  \tikzmath{
    \a = 80;  \AL = 1.5;
    \b = 75; \BL = .3;
    \c = 30;  \CL = 1;
  }
  \coordinate (A) at (\a:\AL*\SignRadius);
  \coordinate (B) at (\b:\BL*\SignRadius);
  \coordinate (C) at (\c:\CL*\SignRadius);
  \coordinate (A') at (\a-180:\AL*\SignRadius);
  \coordinate (B') at (\b-180:\BL*\SignRadius);
  \coordinate (C') at (\c-180:\CL*\SignRadius);
  \fill[c] (A) -- (B) -- (C) -- (A') -- (B') -- (C') -- cycle;
}}
\newcommand{\Lightning}{\usebox{\LightningBox}}

% Phantom icon, useful for alignment.
\newcommand{\PhantomSign}{\phantom{\usebox{\PlusBox}}}

% [mode][name][line spacing]{location}{name}{mod/lines}{epilog}
\NewDocumentCommand{\Diff}{ O{0} O{file} O{5} m m m +m }{

  % Not exactly the corner we expect, but at least the one of first line.
  \AutomaticCoordinates{#2}{#4}

  \begin{scope}[every node/.style={scale=\CodeScale}]
      \foreach \mod/\line [count=\i] in {#6} {\ifcsempty{mod}{}{
        % Define this for \dhi.
        \def\DiffBackgroundCol{\ifdefstring{\mod}{+}{fill=plus}
                              {\ifdefstring{\mod}{-}{fill=minus}
                              {\ifdefstring{\mod}{0}{fill=none}
                              {\ifdefstring{\mod}{m}{fill=modified}
                              {}}}}} % For \dhi.
        \ifnumcomp{\i}{=}{1}{
          % First line has special positionning.
          \node[anchor=base west,\mod,
                alias=#2-line-1,
                alias=last-line,
                alias=#2-last-line,
                alias=widest-line,
                alias=#2-widest-line]
            (line-1) at (#2) {\Code{\line}};
        }{
          \tikzmath{\prev = int(\i - 1);}
          \node[below=#3 of last-line.base west, \mod, anchor=base west,
                alias=#2-line-\i,
                alias=last-line,
                alias=#2-last-line]
                (line-\i) {\Code{\line}};
          % Keep track which line is the widest.
          \tikzmath{
            coordinate \w, \e, \W, \E;
            \w = (last-line.base west);
            \e = (last-line.base east);
            \W = (widest-line.base west);
            \E = (widest-line.base east);
            \d = \ex - \wx;
            \D = \Ex - \Wx;
            \greater = \D < \d;
          }
          \ifdefstring{\greater}{1}{
            \node also [alias=widest-line, alias=#2-widest-line] (last-line);
          }
        }
      }}
  \end{scope}

  % File background.
  \begin{pgfonlayer}{background}
    \coordinate (pad) at (\CodeMargins, \CodeMargins);
    \coordinate (lo) at ($(last-line.south west) - (pad)$);
    \coordinate (up) at ($(widest-line.east |- line-1.north) + (pad)$);
    \node[draw, fill=Light2, fit=(lo)(up), alias=#2-content] (content) {};
  \end{pgfonlayer}

  % Setup file name label.
  \node[above=\FileNameMargins+\CodeMargins of line-1.north west,
        scale=\FileNameScale, anchor=south west, alias=#2-filename-label, #1]
        (filename-label) {\Code{#5}};
  \coordinate (pad) at (\FileNameMargins, \FileNameMargins);
  \coordinate (lower) at ($(filename-label.south west) - (pad)$);
  \coordinate (upper) at ($(filename-label.north east) + (pad)$);
  \node[draw, fit={(lower) (upper)}, alias=#2-filename] (filename) {};
  \begin{pgfonlayer}{background}
    \fill[white] ($(lower) + (0, \eps)$) rectangle (upper);
  \end{pgfonlayer}
  \node[anchor=base east] at ($(filename-label.base west) + (-\SignOffset, .5)$)
    {\ifstrequal{#1}{0}{\PhantomSign}
    {\ifstrequal{#1}{m}{\Tilde}
    {\ifstrequal{#1}{c}{\Lightning}
    {\ifstrequal{#1}{+}{\Plus}{\Minus}}}}};

  % Second pass to append diff signs.
  \foreach \sign/\line [count=\i] in {#6} {\ifcsempty{sign}{}{
    \node[left=\SignOffset of line-\i.base west, anchor=base east]
      (sign-\i)
      {\ifdefstring{\sign}{0}{\PhantomSign}
      {\ifdefstring{\sign}{m}{\Tilde}
      {\ifdefstring{\sign}{c}{\Lightning}
      {\ifdefstring{\sign}{+}{\Plus}
      {\ifdefstring{\sign}{-}{\Minus}{}}}}}};
  }}

  % Construct one total bounding box node.
  \coordinate (lo) at (sign-1.west |- content.south);
  \coordinate (up) at (content.east |- filename.north);
  \node[fit=(lo)(up)] (#2) {};

  % Epilog
  #7

}

% Highlight changed content in diffed line (uses \CurrentMod)
\newcommand{\dhi}[1]{%
  \tikz[baseline={(n.base)}]{%
    \node[fill=Blue1, fill opacity=0.3, text opacity=1,
          scale=1/\CodeScale] (n) {\vphantom{My}\Code{#1}};}%
}

% Draw something looking like a git network.

\tikzmath{
  \CommitRadius = 3;
  \CommitScale = 2.0;
  \CommitMargins = 5;
  \CommitBaseHeight = -2;
  \CommitSpacing = 10;
  \LabelIsep = 2;
  % Trig magic to shape the C.ommit A.rrow so it's evenly thick.
  \CAW = 1.2; % Thickness
  \CAS = 2; % Shorten to not touch commits.
  \CAH = 4; % Head height.
  \CAR = 1.5*\CommitRadius; % Head spread.
  \CAT = \CommitSpacing - 2 * \CAS; % Total arrow tip height.
  \CAC = \CAW * (-4 * \CAH * \CAW + \CAR * sqrt(4*\CAH^2+\CAR^2-4*\CAW^2))
         / (\CAR^2-4*\CAW^2);
  \CAA = asin(\CAW / \CAC); % Angle to vertical.
  \CAB = (\CAR - \CAW) / (2 * tan(\CAA)); % Roof-height under the head.
}
\tikzset{
  commit/.style={line width=2.0, fill=Orange3, draw=Dark3},
  hash/.style={Light5},
  label/.style={draw, line width=1.5, #1, fill=Light3,
                inner sep=\LabelIsep, rounded corners},
  label arrow/.style={-Stealth, line width=1.5, #1,
                      shorten >=3.5*\CommitRadius},
}

% Draw one commit with hash and message.
% [name]{location}{hash}{message}
\NewDocumentCommand{\Commit}{ O{commit-center} m m m }{

  \coordinate (#1) at (#2);
  \path[commit] (#1) circle (\CommitRadius);

  % Hash.
  \node[scale=\CommitScale, anchor=base east, hash]
    at ($(#1) - (\CommitRadius + \CommitMargins, -\CommitBaseHeight)$)
    {\tt #3};

  % Message.
  \node[scale=\CommitScale, anchor=base west, Dark4]
    at ($(#1) + (\CommitRadius + \CommitMargins, \CommitBaseHeight)$)
    {\tt #4};

}

% Chain commits together.
% [anchor][reponame]{location}{hash/message list}
\NewDocumentCommand{\Repo}{ O{center} O{repo} m m }{

  \node[anchor=#1] (#2) at (#3) {\tikz[remember picture]{
    \foreach \hash/\message [count=\i] in {#4} {\ifcsempty{hash}{}{
      \ifnumcomp{\i}{=}{1}{
        \Commit[\hash]{0, 0}{\hash}{\message}
      }{
        \coordinate[above=2*\CommitRadius+\CommitSpacing of previous] (\hash);
        \Commit[\hash]{\hash}{\hash}{\message}

        % Connecting arrow.
        \fill[Light4] ($(previous) + (0, \CommitRadius + \CAS)$)
           -- +(\CAW/2, 0) -- +(\CAW/2, \CAT - \CAH + \CAB)
           -- +(\CAR/2, \CAT - \CAH) -- +(\CAR/2, \CAT - \CAH + \CAC)
           -- +(0, \CAT)
           -- +(-\CAR/2, \CAT - \CAH + \CAC) -- +(-\CAR/2, \CAT - \CAH)
           -- +(-\CAW/2, \CAT - \CAH + \CAB) -- +(-\CAW/2, 0) -- cycle;

      }
      \coordinate (previous) at (\hash);
  }}}};

}

% Pick a commit and point label to it.
% [name][color]{hash}{offset-from-commit}{local-arrow-start}{label-text}
\NewDocumentCommand{\Label}{ O{label} O{Blue4} m m m m }{

  \node[scale=\CommitScale, label=#2] (#1) at ($(#3) + (#4)$) {\tt #6};

  \begin{pgfonlayer}{background}
    \begin{scope}[local to=#1]
      \coordinate (arrow-start) at (#5);
    \end{scope}
    \draw[label arrow=#2] (arrow-start) -- (#3);
  \end{pgfonlayer}

}
% Basic branch label.
% [color]{hash}{offset-from-commit}{local-arrow-start}{branch-name}
\NewDocumentCommand{\Branch}{ O{Blue4} m m m m }{
  \Label[#5][#1]{#2}{#3}{#4}{#5}
}
% Special HEAD label.
% {hash}{offset-from-commit}{local-arrow-start}
\NewDocumentCommand{\Head}{ m m m }{
  \Label[HEAD][Purple4]{#1}{#2}{#3}{HEAD}
  % Also highlight current commit.
  \fill[Yellow1] (#1) circle (.9*\CommitRadius);
}



\newcommand{\TitleText}{<no-title>}
\newcommand{\SubTitleText}{<no-subtitle>}
\newcommand{\PageNumText}{<no-page-number>}
\newcommand{\Progress}{0/1}

\begin{document}%

% Use SLIDE marks for python to easily find all slides.
% Also, this file is parsed lexically in a very brutal way,
% so whitespace and comments are not always insignificant.
% Just use it as a stub so python scripts can bootstrap and generate steps.

% SLIDE Title
\renewcommand{\TitleText}{<notitle>}
\renewcommand{\SubTitleText}{<nosubtitle>}
\renewcommand{\PageNumText}{<nopagenum>}
\Step[bare]{0/0}{

\IntensiveCoordinates{Screen}{c}{0, .5}
\node[scale=12, Dark4] (git) at (c) {\sf \textbf{git}};
\node[below=8 of git, scale=6, Dark4] (fs) {\textbf{from scratch}};
\node[below=12 of fs] (logo) {\PicGitIcon{9cm}{!}};

\IntensiveCoordinates{Screen}{c}{-.95, -.95}
\node[anchor=south west] (gdr) at (c) {\PicGdREcoStat{!}{6cm}};
\IntensiveCoordinates{Screen}{c}{-.15, -.99}
\node[anchor=south] (cesab) at (c) {\PicFRBCESAB{!}{7cm}};
\IntensiveCoordinates{Screen}{c}{+.95, -.85}
\node[anchor=south east] (isem) at (c) {\PicISEM{!}{2.5cm}};

\IntensiveCoordinates{Screen}{c}{-.95, +.97}
\node[anchor=north west] (form) at (c) {\PicFormation{4cm}{!}};

\IntensiveCoordinates{form}{c}{1, .3}
\node[scale=3, right=5 of c, anchor=base west, Dark2] (form-name)
  {Bonnes pratiques pour une recherche reproductible en écologie numérique.};
\node[scale=3, below=13 of form-name.base west, anchor=base west, Dark2]
  {Montpellier, 28 novembre 2022};

\node[scale=3.5, above=13 of isem.north east, anchor=base east, Dark3]
  {Iago Bonnici};

\begin{pgfonlayer}{background}%
  \coordinate[below=5 of form.south] (pad);
  \coordinate (epspad) at (2*\eps, 2*\eps);
  \coordinate (loweps) at ($(Screen.south west) + (epspad)$);
  \coordinate (upeps) at ($(Screen.north east) - (epspad)$);
  \fill[Light2] (loweps|-pad) rectangle (upeps);
  \draw[Light5, line width=1] (loweps|-pad) -- (upeps|-pad);
\end{pgfonlayer}

} % ENDSLIDE

% SLIDE Transition
\renewcommand{\TitleText}{<notitle>}
\renewcommand{\SubTitleText}{<nosubtitle>}
\renewcommand{\PageNumText}{<nopagenum>}
\Step[transition]{0/0}{
} % ENDSLIDE

% SLIDE Clients
\renewcommand{\TitleText}{Git Clients}
\renewcommand{\SubTitleText}{Use the tools you prefer}
\renewcommand{\PageNumText}{1}
\newlength{\U}
\Step[]{1/5}{
  \begin{scope}[inner sep=10]
    \setlength{\U}{75mm}
    \tikzmath{
      \boffset = -5;
    }

    \node[anchor=north, below=5 of Canvas.north] (git)
      {\PicGitLogo{\U}{!}};

    \node[anchor=north, below=20 of git] (console)
      {\PicConsoleGit{!}{\U}};
    \node[below=\boffset of console, scale=\LargeScale]
      {command-line console};

    \node[anchor=east, left=30 of console] (vscode)
      {\PicVSCodeExtension{!}{\U}};
    \node[below=\boffset of vscode, scale=\LargeScale]
      {VSCode extension};

    \node[anchor=west, right=30 of console] (rstudio)
      {\PicRStudioExtension{!}{\U}};
    \node[below=\boffset of rstudio, scale=\LargeScale]
      {RStudio extension};

    \coordinate (tp) at ($(vscode)!.5!(console)$);
    \node[anchor=north, below=65 of tp] (github)
      {\PicGithubLogo{1.5\U}{!}};

    \coordinate (tp) at ($(console)!.5!(rstudio)$);
    \node[anchor=north] (gitlab) at (github -| tp)
      {\PicGitlabLogo{1.5\U}{!}};

    \begin{scope}[every path/.style={-Stealth, line width=2, Dark4}]
      \draw (vscode.north) -- (git.west);
      \draw (console.north) -- (git.south);
      \draw (rstudio.north) -- (git.east);
      \draw (github.north) .. controls +(0, 100) .. (git.south west);
      \draw (gitlab.north) .. controls +(0, 130) .. (git.south east);
    \end{scope}

    \HighlightShade{git}

  \end{scope}
} % ENDSLIDE

% SLIDE Pizzas
\renewcommand{\TitleText}{The Pizzas Repository}
\renewcommand{\SubTitleText}{Crafting your First Commits}
\renewcommand{\PageNumText}{2}
\Step[]{2/5}{

  \FileTree[files]{-1, 1}{
    folder/+/root/rootfolder,
  }

  \Diff[m][north east][diff][5]{1, 1}{file.ext}{
    +/{One line},
  }{}

  \Repo[repo][simple][1]{-1, -1}{}{}

  \Command[base][][.5][5][]{0, 0}{git init}

} % ENDSLIDE

% SLIDE Staging
\renewcommand{\TitleText}{Constructing a Commit}
\renewcommand{\SubTitleText}{The Whole Process}
\renewcommand{\PageNumText}{3}
\tikzmath{
  \CommitSpacingSafe = \CommitSpacing;
  \CommitSpacing = 184; % Highjack locally to better see.
}
\tikzset{
  area/.style 2 args={fill=#1, draw=#2,
                      line width=2,
                      fill opacity=.6,
                      minimum width=250mm,
                      minimum height=46mm,
                      alias=highest,
                      anchor=south,
                      },
  area label/.style={scale=\LargeScale, anchor=base west, Dark3,
                     right=3 of highest.west},
  machine/.style={line width=2, draw=Dark1, fill=Light3},
  machine label/.style={anchor=base west, scale=2.4, Dark3},
  not left/.style={right=#1, anchor=west}, % -_-"
  not right/.style={left=#1, anchor=east},
}
\NewDocumentCommand{\MakeArea}{ O{highest.north} m m m m }{
  \AutomaticCoordinates{c}{#1}
  \node[area={#2}{#3}] (#4) at (c) {};
  \node[area label] (#4-label) {#5};
  % Provide adjusted coordinates for lines match.
  \coordinate[above=3*\eps of #4.south east] (#4-s);
  \coordinate[below=3*\eps of #4.north east] (#4-e);
}
% [slide][crit][labeled][offset]{side}{start}{end}{text}
\NewDocumentCommand{\SwitchArrow}{ O{.35} O{.35} O{0} O{5} m m m m }{{
  \ifstrequal{#5}{left}{
    \IntensiveCoordinates{#6}{s}{-#1, -.2}
    \IntensiveCoordinates{#7}{e}{-#1+.10, -.2}
  }{
    \IntensiveCoordinates{#6}{s}{#1+.10, .23}
    \IntensiveCoordinates{#7}{e}{#1, .23}
  }
  \begin{pgfonlayer}{command-background}
  \begin{scope}[transparency group, opacity=0.7]
    \draw[-Stealth, Brown3, line width=10]
      (s) to[bend left=25]
      % Coordinate escapes the group.
      node[name=n, not #5=#4, pos=#2, anchor=center] {} (e);
  \end{scope}
  \end{pgfonlayer}
  \ifstrequal{#3}{0}{}{
  \begin{pgfonlayer}{command-text}
    \node[command text, scale=.7, not #5, anchor=center] (n) at (n) {\Code{#8}};
  \end{pgfonlayer}
  \begin{pgfonlayer}{command-background}
    \coordinate (pad) at (1, 1);
    \coordinate (low) at ($(n.south west) - (pad)$);
    \coordinate (up) at ($(n.north east) + (pad)$);
    \path[command box, line width=1, fill=Light2] (low) rectangle (up);
  \end{pgfonlayer}
  }
}}
\Step[]{3/5}{

\Repo[repo][simple][1]{-1, -1}{}{}

\MakeArea[.17, -.9]{Blue1}{Blue3}{last}{in Commit}
\MakeArea{Purple1}{Purple3}{editor}{\hspace{-.2em}\it<in editor>}
\MakeArea{Yellow1}{Yellow5}{modified}{Modified}\node[area label, below=1 of modified-label] {(*)};
\MakeArea{Green1}{Green3}{stage}{Stage}\node[area label, below=1 of stage-label] {(*)};
\MakeArea{Blue1}{Blue3}{next}{in Commit}

\coordinate (right) at (Canvas.east);
\coordinate[left=6 of right] (left);
\path[machine] (last-s) rectangle (next-e -| right);
\node[machine label, right=6 of stage.east] {on disk};

\path[machine, fill=white] (editor-s) rectangle (editor-e -| left);
\node[machine label, right=2 of editor.east] {in RAM};

\IntensiveCoordinates{last}{c}{.35, 0}
\FileTree[unit-tree]{c}{file/0/filenode/filename.ext}

\SwitchArrow[.49][.35][1][0]{left}{last}{editor}{<keyboard>}
\SwitchArrow[.54][.35][1][0]{left}{editor}{modified}{<ctrl-S>}
\SwitchArrow[.59][.35][1][0]{left}{modified}{stage}{\$ git \gkw{add}}
\SwitchArrow[.67][.35][1][0]{left}{stage}{next}{\$ git \gkw{commit}}
\SwitchArrow[.60][.35][1][22]{right}{next}{stage}{\$ git reset \CommandHighlight{Green1}{--soft}}
\SwitchArrow[.70][.40][1][20]{right}{stage}{modified}{\$ git \gkw{reset}}
\SwitchArrow[.80][.35][0][0]{right}{next}{last}{}
\SwitchArrow[.80][.35][0][0]{right}{stage}{last}{}
\SwitchArrow[.80][.19][1][25]{right}{modified}{last}{\$ git reset \CommandHighlight{Red1}{--hard}}
\SwitchArrow[.60][.35][1][1]{right}{editor}{last}{<ctrl-Z>}

\Diff[+][north east][gitignore][8]{left=3 of editor.north west}{.gitignore}{
  +/{\# Files to ignore:},
}{}

} % ENDSLIDE

% SLIDE Remote
\renewcommand{\TitleText}{Share Your Project}
\renewcommand{\SubTitleText}{Creating a Remote Repository}
\renewcommand{\PageNumText}{4}
\tikzmath{
  % \CommitSpacing = \CommitSpacingSafe; % DEBUG while selecting only this slide.
  \CommitSpacing = 11;
}
\Step[]{4/5}{

  \FileTree[myfiles]{-1, 1}{
    folder/+/root/rootfolder,
  }

  \FileTree[theirfiles]{.55, 1}{
    folder/+/root/rootfolder,
  }

  \begin{scope}[local to=Canvas]\begin{pgfonlayer}{background}
    \node[opacity=.15] (github) at (0, .5) {\PicGithubLogo{10cm}{!}};
    \node[opacity=.15] (mymachine) at (-.7, -.6) {\PicMyMachine{75mm}{!}};
    \node[opacity=.15] (theirmachine) at (+.7, -.6) {\PicTheirMachine{75mm}{!}};
  \end{pgfonlayer}\end{scope}

  \Repo[mine][mixed][1]{-1, -1}{}{}


  \Repo[remote][mixed][1]{-1, -1}{}{}


  \Repo[theirs][mixed][1]{-1, -1}{}{}

} % ENDSLIDE

% SLIDE Conflicts
\renewcommand{\TitleText}{Resolve Conflicts}
\renewcommand{\SubTitleText}{What a ``Conflict'' Means}
\renewcommand{\PageNumText}{5}
\tikzset{
  zone/.style={scale=\LargeScale, Dark4},
  zone code/.style={scale=2.3},
}
\Step[]{5/5}{

\AutomaticCoordinates{low}{-.34, -.95}
\AutomaticCoordinates{up}{+.34, +.95}
\AutomaticCoordinates{lexical}{0, -.1}
\AutomaticCoordinates{semantic}{0, .3}
\coordinate (mid) at ($(up)!.5!(low)$);
\coordinate (bottom) at (low-|mid);

\path[fill=Blue1, draw=Blue3, line width=2, fill opacity=1] (low) rectangle (up);

\path[fill=Brown1, draw=Brown3, line width=2, fill opacity=0.5]
  (lexical) ellipse (63 and 70);

\path[fill=Purple1, draw=Purple3, line width=2, fill opacity=0.5]
  (semantic) ellipse (63 and 70);

\node[zone, anchor=south, above=27 of bottom] (line) {no conflicts};
\node[zone, below=3 of line] (line) {\it (happy zone)};
\node[zone, below=1 of line, Green5] (line) {\rotatebox{-90}{\Code{:)}}};

\node[zone, below=25 of lexical] (line) {\bf lexical conflicts};
\node[zone code, below=3 of line] (line) {\Code{>> git conflicts <<}};
\node[zone, below=1 of line, Red4] (line) {\rotatebox{-90}{\Code{:(}}};

\node[zone, above=44 of semantic] (line) {\bf semantic conflicts};
\node[zone code, below=3 of line] (line) {\Code{>> git sees not <<}};
\node[zone, below=-2 of line, Red4] (line) {\rotatebox{-90}{\Code{:(}}};

\node[zone code] at ($(lexical)!.55!(semantic)$) (line) {\Code{>>git's got your back<<}};
\node[zone, below=3 of line] (line) {\it (happy zone)};
\node[zone, below=1 of line, Green5] (line) {\rotatebox{-90}{\Code{:)}}};

\Diff[m][north west][left][8]{-1, .8}{MY\_VERSION}{
  0/{my line},
}{}

\Diff[m][north east][right][8]{1, .8}{THEIR\_VERSION}{
  0/{their line},
}{}

\Diff[0][north][merge][8]{$(left.north)!.5!(right.north)$}{MERGED\_VERSION}{
  0/{merged line},
}{}

\AutomaticCoordinates{c}{Canvas.center}
\node[scale=\LargeScale, Dark4] (message) at (c) {\bf message};
\coordinate[below=1 of message] (c);
\draw[line width = 2, Dark4] (c-|message.west) -- (c-|message.east);

} % ENDSLIDE

\end{document}

