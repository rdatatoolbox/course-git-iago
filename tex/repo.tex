% Draw something looking like a git network.
% Only 2 parallel git chains supported now.

\tikzmath{
  \CommitRadius = 3;
  \CommitScale = 1.8;
  \CommitMargins = 5;
  \CommitBaseHeight = -2;
  \CommitSpacing = 11;
  \BranchesSpacing = 9;
  \LabelIsep = 2;
  \CommitArrowShorten = \CommitRadius + 2;
  \LabelArrowShorten = \CommitRadius + 1.5;
  \CommandScale = 3.5;
  \CommandPad = 2;
}
\tikzset{
  commit/.style={line width=2.0, fill=Orange3, draw=Dark3},
  hash/.style={Light5},
  commit arrow/.style={-{Straight Barb[angle'=70, length=3mm]},
                       Light5, line width=4.5,
                       shorten >=\CommitArrowShorten mm,
                       shorten <=\CommitArrowShorten mm,
                       },
  label/.style={line width=1.5, draw, #1, fill=Light3, scale=\CommitScale,
                inner sep=\LabelIsep, rounded corners, -Stealth},
  label hi/.style={label=#1, fill=Yellow1},
  label hi ring/.style={label hi=#1}, % Same but with an extra ring.
  %
  repo label/.style={scale=\NormalScale, inner sep=2, draw, Dark3, fill=Light2},
  remote label/.style={repo label, Blue4, fill=Light2},
  %
  remote pointer/.style={-Stealth, line width=2, Brown2, dashed},
  remote flow/.style={remote pointer, solid, line width=20},
  remote arrow label/.style={scale=1.8, above=2, sloped,
                             draw, solid, inner sep = 2},
  remote arrow label hi/.style={remote arrow label, fill=Yellow1},
  %
  command text/.style={scale=\CommandScale, Dark4},
  command text error/.style={command text, Red2},
  command text ok/.style={command text, Green5},
  command text fade/.style={command text, opacity=0.2},
  command box/.style={draw=Dark3, fill=Light4, line width=2.5},
  command box error/.style={command box, draw=Red2, fill=Light3},
  command box ok/.style={command box, draw=Green5, fill=Light3},
  command box fade/.style={command box, opacity=0.2},
  url/.style={fill=Light2},
  url hi/.style={url, fill=Yellow1},
}

\newsavebox{\Lock}
\savebox{\Lock}{\tikz[baseline={(4.5, 1.6)}]{
  \tikzmath{ \W=8; \H=8; \h=6; }
  \fill[Dark3, rounded corners=7] (0, 0) rectangle (\W, \h);
  \draw[Dark3, rounded corners=6, line width=5]
    (.25*\W, 0.5*\H) rectangle (.75*\W, \H);
  \fill[Light4, rounded corners=2] (.25*\W, .28*\h) rectangle (.75*\W, .74*\h);
  \draw[Dark3, line width=2] (.5*\W, .35*\h) -- +(0, .33*\h);
}}

% Draw one commit with hash and message, use [offset] to align with 2 branches.
% [type][hash-offset][message-offset][type][index]{location}{hash}{message}
\NewDocumentCommand{\Commit}{ O{} O{0} O{0} O{} O{0} m m m }{
  \coordinate[alias=\LatestRepo-#7,
              alias=#5,
              alias=\LatestRepo-#5] (#7) at (#6);
  \path[commit] (#7) circle (\CommitRadius);

  % Hash.
  \node[scale=\CommitScale, anchor=base east, hash,
        alias=\LatestRepo-#7-hash,
        alias=#5-hash,
        alias=\LatestRepo-#5-hash,
        ] (#7-hash)
    at ($(#7) - (\CommitRadius + \CommitMargins + #2, -\CommitBaseHeight)$)
    {\Code{#7}};

  % Message.
  \node[scale=\CommitScale, anchor=base west, Dark4,
        alias=\LatestRepo-#7-message,
        alias=#5-message,
        alias=\LatestRepo-#5-message,
        ] (#7-message)
    at ($(#7) + (\CommitRadius + \CommitMargins - #2 + #3, \CommitBaseHeight)$)
    {\Code{#8}};

  \IfSubStr{#4}{hi}{\IfSubStr{#4}{fade}{\def\op{.2}}{\def\op{1}}}{\def\op{0}}
  \HighlightSquare[.4][\op]{#7-message.south west}{#7-message.north east}
}

% Chain commits together.
% Aligning commit messages is either:
%   - simple: all aligned with room for 1 commits chain.
%   - double: all aligned with room for 2 commits chain.
%   - mixed: eich message is aligned to its own chain.
% Location given is the one of the first commit (disk center).
% [reponame][alignment][opacity]{location}{type/hash/message list}{labels}
\NewDocumentCommand{\Repo}{ O{repo} O{simple} O{1} m m +m }{

  % Update this global variable so commits, branches etc.
  % can be aliased with (reponame-nodename) in addition to just (nodename).
  \def\LatestRepo{#1}

  \AutomaticCoordinates{#1}{#4}
  \begin{scope}[transparency group, opacity=#3]
    \def\merged{1} % Lower when there are two chains.
    % Vertical/Horizontal commit spacing.
    \tikzmath{
      \V = 2*\CommitRadius + \CommitSpacing;
      \H = \BranchesSpacing;
      \Hoff = 0;
    }
    % All messages need to be offset by 1 unit if the repo is double
    % so they remain aligned.
    \ifstrequal{#2}{double}{\tikzmath{ \Hoff = \H; }}{}
    \foreach \type/\hash/\message [count=\i] in {#5} {\ifcsempty{hash}{}{
      \IfSubStr{\type}{fade}{\begin{scope}[transparency group, opacity=.3]}{}
      \ifnumcomp{\i}{=}{1}{
        \Commit[\hash][0][\Hoff][\type][1]{#1}{\hash}{\message}
        % Keep track of last commit on the straight/parallel chain..
        \coordinate (straight) at (\hash);
        \coordinate (parallel) at (\hash); % (but they start merged)
        \coordinate (last) at (\hash); % .. and of last commit at all.
      }{
        \coordinate (\hash) at ($(straight |- last) + (\H, \V)$);
        % Commit position depends on its type.
        % I: commit on the straight chain.
        % Y: fork commit (the first on the parallel chain).
        % H: commit on the parallel chain.
        % A: merge commit (on the straight chain).
        \IfSubStr{\type}{Y}{
          \coordinate (parallel) at (straight);
        }{}
        \ifboolexpr{ test {\IfSubStr{\type}{H}}
                  or test {\IfSubStr{\type}{Y}} }{
          \ifdefstring{\merged}{1}{
            \draw[commit arrow] (parallel) -- (\hash);
          }{
            \coordinate
              [above=\CommitRadius + .5*\CommitSpacing of parallel] (c);
            \draw[commit arrow] (parallel) .. controls (\hash |- c) .. (\hash);
          }
          \coordinate (parallel) at (\hash);
          \def\merged{0}
        }{
          \coordinate (\hash) at (straight |- \hash);
          \draw[commit arrow] (straight) -- (\hash);
          \coordinate (straight) at (\hash);
          \tikzmath{ \H = 0; }
        }
        \ifstrequal{#2}{mixed}{\tikzmath{\Hoff = \H;}}{}
        \Commit[\type][\H][\Hoff][\type][\i]{\hash}{\hash}{\message}
        \IfSubStr{\type}{A}{%
          \coordinate[below=\CommitRadius + .5*\CommitSpacing of \hash] (c);
          \draw[commit arrow] (parallel)
                  .. controls (parallel |- c) .. (\hash);
          \coordinate (straight) at (\hash);
          \coordinate (parallel) at (\hash);
          \def\merged{1}
        }{}
    }
    \coordinate[alias=#1-last] (last) at (\hash);
    \node also[alias=last-hash, alias=#1-last-hash] (\hash-hash);
    \node also[alias=last-message, alias=#1-last-message] (\hash-message);
    \IfSubStr{\type}{fade}{\end{scope}}{}
  }}
  % Labels go here, within the embed tikz picture,
  % so they are calculated in the global repo dimensions.
  #6
  \end{scope}
}

% Pick a commit and point label to it.
% If 'base' appears in the ref,
% it is assumed that horizontal alignment is needed,
% so the local arrow 'start' is interpreted
% as an absolute vertical offset from the base,
% the reference end destination is given the same offset,
% and the arrow is not shortened.
% (Useful when eg. HEAD is pointing to a branch.)
% If local arrow start is 'noarrow', there is no pointer.
% (Useful on empty repos or to stick origin/main to main.)
% Special text '<branch>-lock' is interpreted to produce a lock icon,
% with the same positionning logic as other labels.
% [name][anchor][style]{ref}{offset}{start}{text}
\NewDocumentCommand{\Label}{ O{unnamed} O{base} O{} m m m m }{

  \IfSubStr{#4}{base}{\IfSubStr{#4}{west}
   {\coordinate[left=#5 of #4] (c);}
   {\coordinate[right=#5 of #4] (c);}}
   {\coordinate (c) at ($(#4) + (#5)$);}

  \IfSubStr{#1}{-lock}{%
    \node[inner sep=4, anchor=#2, alias=\LatestRepo-#1, inner sep=4]
      (#1) at (c) {\usebox{\Lock}};
  }{
    \node[label #3, anchor=#2, alias=\LatestRepo-#1, name path=rec]
      (#1) at (c) {\Code{#7}};
  }

  \IfSubStr{#3}{ring}{
    \HighlightSquareRing[1]{#1}
  }{
    \HighlightSquareRing[1][0]{#1} % Phantom to not shift when highlighting.
  }

  \ifstrequal{#6}{noarrow}{}{
    \IfSubStr{#4}{base}{
      % Assume we're pointing to another label, horizontally.
      \IfSubStr{#4}{west}{
        \coordinate (s) at ($(#1.base east) + (-.5, #6)$);
      }{
        \coordinate (s) at ($(#1.base west) + (+.5, #6)$);
      }
      \coordinate[above=#6 of #4] (e);
      \draw[label #3, shorten >=0] (s) -- (e);
    }{
      % Assume we're pointing to a commit.
      \coordinate (s) at ($(#1.base west) + (#6)$);
      % Calculate intersection to correctly clip the arrow.
      \path[name path=arrow] (s) -- (#4);
      \path[name intersections={of=rec and arrow, by=cross}];
      \draw[label #3, shorten >=\LabelArrowShorten mm] (cross) -- (#4);
    }
  }

}

% WARNING: \Branch, \RemoteBranch, \Head
% are left here to ease direct use within LaTeX,
% but the python code directly generates \Label commands now.
% Basic branch label.
% [anchor][style]{ref}{offset}{start}{name}
\NewDocumentCommand{\Branch}{ O{base} O{} m m m m }{
  \Label[#6][#1][#2=Blue4]{#3}{#4}{#5}{#6}
}
% Remote branch label.
% [anchor][style]{ref}{offset}{start}{name}
\NewDocumentCommand{\RemoteBranch}{ O{base} O{} m m m m }{
  \Label[#6][#1][#2=Brown2]{#3}{#4}{#5}{#6}
}
% HEAD label.
% [anchor][style]{ref}{offset}{start}
\NewDocumentCommand{\Head}{ O{base} O{} m m m }{
  \Label[HEAD][#1][#2=Purple4]{#3}{#4}{#5}{HEAD}
}

% Highlight current commit.
% {hash}
\newcommand{\HighlightCommit}[1]{
  \fill[Yellow1] (#1) circle (.9*\CommitRadius);
}

% {anchor}{location}{label}
\newcommand{\LocalRepoLabel}[3]{%
  \IntensiveCoordinates{Canvas}{c}{#2}
  \node[repo label, anchor=#1] at (c) {\textit{#3}};
}
% {style}{text}
\newcommand{\UrlHighlight}[2]{%
  \tikz[baseline={(n.base)},inner sep=.5]{\node[#1](n){\vphantom{My}#2};}%
}
% [highlight]{anchor}{location}{account}{name}
\NewDocumentCommand{\RemoteRepoLabel}{ O{} m m m m }{{%
  \def\account{\IfSubStr{#1}{account}
    {\UrlHighlight{url hi}{#4}}
    {\UrlHighlight{url}{#4}}}
  \def\name{\IfSubStr{#1}{name}
    {\UrlHighlight{url hi}{#5}}
    {\UrlHighlight{url}{#5}}}
  \AutomaticCoordinates{c}{#3}
  \node[remote label, anchor=#2] at (c) {\Code{https://gitlab.com/\account/\name}};
}}


% Arrow from one repo to another.
% [name][bend][side][style]{start}{end}
\NewDocumentCommand{\RemoteArrow}{ O{} O{0} O{left} O{} m m }{
  \AutomaticCoordinates{s}{#5}
  \AutomaticCoordinates{e}{#6}
  \ifstrequal{#1}{}{
    \begin{scope}[transparency group, opacity=.5]
      \draw[remote flow] (s) to[bend #3=#2] (e);
    \end{scope}
  }{
    \draw[remote pointer] (s) to[bend #3=#2]
      node[remote arrow label #4] {\Code{#1}} (e);
  }
}

% Display the command used.
% If 'start' is given, make it like a cartoon bubble,
% pointing to intensive horizontal coordinate 'end'.
% [anchor][start][end][aperture][style]{location}{text}
\NewDocumentCommand{\Command}{ O{base} O{} O{.5} O{5} O{} m m }{
  \begin{pgfonlayer}{command-text}
    \AutomaticCoordinates{c}{#6}
    \node[command text #5, anchor=#1] (command) at (c) {\Code{\$ #7}};
    \coordinate (pad) at (\CommandPad, \CommandPad);
    \coordinate (low) at ($(command.south west) - (pad)$);
    \coordinate (up) at ($(command.north east) + (pad)$);
  \end{pgfonlayer}
  \begin{pgfonlayer}{command-background}
    \ifstrequal{#2}{}{
      \path[command box #5] (low) rectangle (up);
    }{
      \AutomaticCoordinates{s}{#2}
      \coordinate (e) at ($(command.west)!#3!(command.east)$);
      \tikzmath{
        coordinate \comm, \start;
        \comm = (c);
        \start = (s);
        \IsStartHigher = \starty > \commy;
      }
      \ifdefstring{\IsStartHigher}{1}{\def\corner{up}}{\def\corner{low}}
      \coordinate (p) at (e |- \corner);
      \coordinate[left=.5*#4 of p] (e1);
      \coordinate[right=.5*#4 of p] (e2);
      \ifdefstring{\IsStartHigher}{1}{
        \path[command box #5]
          (s) -- (e1) -- (low |- up) --
          (low) -- (low -| up) -- (up) -- (e2) -- cycle;
      }{
        \path[command box #5]
          (s) -- (e1) -- (low) --
          (low |- up) -- (up) -- (up |- low) -- (e2) -- cycle;
      }
    }
  \end{pgfonlayer}
}

% Highlight command parts.
% [vphantom]{color}{text}
\NewDocumentCommand{\CommandHighlight}{ O{Ip} m m }{%
  \tikz[baseline={(n.base)}]
    {\node[fill=#2](n){\vphantom{#1}\Code{#3}};}%
}
% Git keywords.
\NewDocumentCommand{\gkw}{ O{Ip} m }{\CommandHighlight[#1]{Blue1}{#2}}
% Just highlight one word in particular..
\NewDocumentCommand{\ghi}{ O{Ip} m }{\CommandHighlight[#1]{Light3}{#2}}
