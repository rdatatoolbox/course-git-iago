% Draw something looking like a git network.

\tikzmath{
  \CommitRadius = 3;
  \CommitScale = 2.0;
  \CommitMargins = 5;
  \CommitBaseHeight = -2;
  \CommitSpacing = 10;
  \LabelIsep = 2;
  % Trig magic to shape the C.ommit A.rrow so it's evenly thick.
  \CAW = 1.2; % Thickness
  \CAS = 2; % Shorten to not touch commits.
  \CAH = 4; % Head height.
  \CAR = 1.5*\CommitRadius; % Head spread.
  \CAT = \CommitSpacing - 2 * \CAS; % Total arrow tip height.
  \CAC = \CAW * (-4 * \CAH * \CAW + \CAR * sqrt(4*\CAH^2+\CAR^2-4*\CAW^2))
         / (\CAR^2-4*\CAW^2);
  \CAA = asin(\CAW / \CAC); % Angle to vertical.
  \CAB = (\CAR - \CAW) / (2 * tan(\CAA)); % Roof-height under the head.
}
\tikzset{
  commit/.style={line width=2.0, fill=Orange3, draw=Dark3},
  hash/.style={Light5},
  label/.style={draw, line width=1.5, #1, fill=Light3,
                inner sep=\LabelIsep, rounded corners},
  label arrow/.style={-Stealth, line width=1.5, #1,
                      shorten >=3.5*\CommitRadius},
}

% Draw one commit with hash and message.
% [name]{location}{hash}{message}
\NewDocumentCommand{\Commit}{ O{commit-center} m m m }{

  \coordinate (#1) at (#2);
  \path[commit] (#1) circle (\CommitRadius);

  % Hash.
  \node[scale=\CommitScale, anchor=base east, hash]
    at ($(#1) - (\CommitRadius + \CommitMargins, -\CommitBaseHeight)$)
    {\tt #3};

  % Message.
  \node[scale=\CommitScale, anchor=base west, Dark4]
    at ($(#1) + (\CommitRadius + \CommitMargins, \CommitBaseHeight)$)
    {\tt #4};

}

% Chain commits together.
% [anchor][reponame]{location}{hash/message list}
\NewDocumentCommand{\Repo}{ O{center} O{repo} m m }{

  \node[anchor=#1] (#2) at (#3) {\tikz[remember picture]{
    \foreach \hash/\message [count=\i] in {#4} {\ifcsempty{hash}{}{
      \ifnumcomp{\i}{=}{1}{
        \Commit[\hash]{0, 0}{\hash}{\message}
      }{
        \coordinate[above=2*\CommitRadius+\CommitSpacing of previous] (\hash);
        \Commit[\hash]{\hash}{\hash}{\message}

        % Connecting arrow.
        \fill[Light4] ($(previous) + (0, \CommitRadius + \CAS)$)
           -- +(\CAW/2, 0) -- +(\CAW/2, \CAT - \CAH + \CAB)
           -- +(\CAR/2, \CAT - \CAH) -- +(\CAR/2, \CAT - \CAH + \CAC)
           -- +(0, \CAT)
           -- +(-\CAR/2, \CAT - \CAH + \CAC) -- +(-\CAR/2, \CAT - \CAH)
           -- +(-\CAW/2, \CAT - \CAH + \CAB) -- +(-\CAW/2, 0) -- cycle;

      }
      \coordinate (previous) at (\hash);
  }}}};

}

% Pick a commit and point label to it.
% [name][color]{hash}{offset-from-commit}{local-arrow-start}{label-text}
\NewDocumentCommand{\Label}{ O{label} O{Blue4} m m m m }{

  \node[scale=\CommitScale, label=#2] (#1) at ($(#3) + (#4)$) {\tt #6};

  \begin{pgfonlayer}{background}
    \begin{scope}[local to=#1]
      \coordinate (arrow-start) at (#5);
    \end{scope}
    \draw[label arrow=#2] (arrow-start) -- (#3);
  \end{pgfonlayer}

}
% Basic branch label.
% [color]{hash}{offset-from-commit}{local-arrow-start}{branch-name}
\NewDocumentCommand{\Branch}{ O{Blue4} m m m m }{
  \Label[#5][#1]{#2}{#3}{#4}{#5}
}
% Special HEAD label.
% {hash}{offset-from-commit}{local-arrow-start}
\NewDocumentCommand{\Head}{ m m m }{
  \Label[HEAD][Purple4]{#1}{#2}{#3}{HEAD}
  % Also highlight current commit.
  \fill[Yellow1] (#1) circle (.9*\CommitRadius);
}

