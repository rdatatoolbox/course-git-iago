% Draw something looking like a git network.
% Only 2 parallel git chains supported now.

\tikzmath{
  \CommitRadius = 3;
  \CommitScale = 2.0;
  \CommitMargins = 5;
  \CommitBaseHeight = -2;
  \CommitSpacing = 10;
  \BranchesSpacing = 8;
  \LabelIsep = 2;
  \CommitArrowShorten = \CommitRadius + 2;
  \LabelArrowShorten = \CommitRadius + 2;
  \CommandScale = 4;
}
\tikzset{
  commit/.style={line width=2.0, fill=Orange3, draw=Dark3},
  hash/.style={Light5},
  commit arrow/.style={-{Straight Barb[angle'=70, length=3mm]},
                       Light5, line width=4.5,
                       shorten >=\CommitArrowShorten mm,
                       shorten <=\CommitArrowShorten mm,
                       },
  label/.style={draw, line width=1.5, #1, fill=Light3,
                inner sep=\LabelIsep, rounded corners},
  label-hi/.style={label=#1, fill=Yellow1},
  label arrow/.style={-Stealth, line width=1.5, #1,
                      shorten >=\LabelArrowShorten mm},
  repo label/.style={scale=\NormalScale, inner sep=2, draw=Light5, fill=Light2}
}

% Draw one commit with hash and message, use [offset] to align with 2 branches.
% [name][hash-offset][messa-offset]{location}{hash}{message}
\NewDocumentCommand{\Commit}{ O{commit-center} O{0} O{0} m m m }{

  \coordinate (#1) at (#4);
  \path[commit] (#1) circle (\CommitRadius);

  % Hash.
  \node[scale=\CommitScale, anchor=base east, hash] (#5-hash)
    at ($(#1) - (\CommitRadius + \CommitMargins + #2, -\CommitBaseHeight)$)
    {\tt #5};

  % Message.
  \node[scale=\CommitScale, anchor=base west, Dark4] (#5-message)
    at ($(#1) + (\CommitRadius + \CommitMargins - #2 + #3, \CommitBaseHeight)$)
    {\tt #6};

}

% Chain commits together.
% [anchor][reponame][simple|double]{location}{type/hash/message list}
\NewDocumentCommand{\Repo}{ O{center} O{repo} O{simple} m m }{


  \IntensiveCoordinates{Canvas}{loc}{#4}
  \node[anchor=#1] (#2) at (loc) {\tikz[remember picture]{
    \def\merged{1} % Lower when there are two chains.
    % Vertical/Horizontal commit spacing.
    \tikzmath{
      \V = 2*\CommitRadius + \CommitSpacing;
      \H = \BranchesSpacing;
      \Hoff = 0;
    }
    % All messages need to be offset by 1 unit if the repo is double
    % so they remain aligned.
    \ifstrequal{#3}{double}{\tikzmath{ \Hoff = \H; }}{}
    \foreach \type/\hash/\message [count=\i] in {#5} {\ifcsempty{hash}{}{
      \ifnumcomp{\i}{=}{1}{
        \Commit[\hash][0][\Hoff]{0, 0}{\hash}{\message}
        % Keep track of last commit on the straight/parallel chain..
        \coordinate (straight) at (\hash);
        \coordinate (parallel) at (\hash); % (but they start merged)
        \coordinate (last) at (\hash); % .. and of last commit at all.
      }{
        \coordinate (\hash) at ($(straight |- last) + (\H, \V)$);
        % Commit position depends on its type.
        % I: commit on the straight chain.
        % Y: fork commit (the first on the parallel chain).
        % H: commit on the parallel chain.
        % A: merge commit (on the straight chain).
        \ifdefstring{\type}{Y}{
          \coordinate (parallel) at (straight);
        }{}
        \ifboolexpr{ test {\ifdefstring{\type}{H}}
                  or test {\ifdefstring{\type}{Y}} }{
          \ifdefstring{\merged}{1}{
            \draw[commit arrow] (parallel) -- (\hash);
          }{
            \coordinate
              [above=\CommitRadius + .5*\CommitSpacing of parallel] (c);
            \draw[commit arrow] (parallel) .. controls (\hash |- c) .. (\hash);
          }
          \coordinate (parallel) at (\hash);
          \def\merged{0}
        }{
          \coordinate (\hash) at (straight |- \hash);
          \draw[commit arrow] (straight) -- (\hash);
          \coordinate (straight) at (\hash);
          \tikzmath{ \H = 0; }
        }
        \Commit[\hash][\H][\Hoff]{\hash}{\hash}{\message}
        \ifdefstring{\type}{A}{%
          \coordinate[below=\CommitRadius + .5*\CommitSpacing of \hash] (c);
          \draw[commit arrow] (parallel)
                  .. controls (parallel |- c) .. (\hash);
          \coordinate (straight) at (\hash);
          \coordinate (parallel) at (\hash);
          \def\merged{1}
        }{}
    }
    \coordinate (last) at (\hash);
  }}}};

}

% Pick a commit and point label to it.
% If 'base' appears in the ref,
% it is assumed that horizontal alignment is needed,
% so the local arrow start is interpreted
% as an absolute vertical offset from the base,
% the reference end destination is given the same offset,
% and the arrow is not shortened.
% (Useful when HEAD is pointing to a branch.)
% If local arrow start is 'noarrow', there is no pointer.
% (Useful on empty repos.)
% [name][color][anchor][style]
% {ref}{offset-from-commit}{local-arrow-start}{label-text}
\NewDocumentCommand{\Label}{ O{label} O{Blue4} O{base} O{label} m m m m }{

  \node[scale=\CommitScale, #4=#2, anchor=#3]
    (#1) at ($(#5) + (#6)$) {\tt #8};

  \ifstrequal{#4}{label-hi}{
    \coordinate (pad) at (1, 1);
    \coordinate (lo) at ($(#1.south west) - (pad)$);
    \coordinate (up) at ($(#1.north east) + (pad)$);
    \draw[Yellow1, line width=4] (lo) rectangle (up);
  }{}

  \ifstrequal{#7}{noarrow}{}{
    \begin{pgfonlayer}{background}
      \IfSubStr{#5}{base}{
        \IfSubStr{#5}{west}{
          \coordinate (s) at ($(#1.base east) + (-1, #7)$);
        }{
          \coordinate (s) at ($(#1.base west) + (+1, #7)$);
        }
        \coordinate[above=#7 of #5] (e);
        \draw[label arrow=#2, shorten >=0] (s) -- (e);
      }{
        \begin{scope}[local to=#1]
          \coordinate (s) at (#7);
        \end{scope}
        \draw[label arrow=#2] (s) -- (#5);
      }
    \end{pgfonlayer}
  }

}
% Basic branch label.
% offset: position wrt commit position
% local: starting point of the arrow in local coordinates.
% [color][anchor][style]{hash}{offset}{local}{name}
\NewDocumentCommand{\Branch}{ O{Blue4} O{base} O{label} m m m m }{
  \Label[#7][#1][#2][#3]{#4}{#5}{#6}{#7}
}
% Same for HEAD label.
% [anchor][style]{ref}{offset}{local}
\NewDocumentCommand{\Head}{ O{base} O{label} m m m }{
  \Label[HEAD][Purple4][#1][#2]{#3}{#4}{#5}{HEAD}
}
% Highlight current commit.
% {hash}
\newcommand{\HighlightCommit}[1]{
  \fill[Yellow1] (#1) circle (.9*\CommitRadius);
}

% {anchor}{location}{label}
\newcommand{\LocalRepoLabel}[3]{%
  \IntensiveCoordinates{Canvas}{c}{#2}
  \node[repo label, anchor=#1] at (c) {#3};
}
% {anchor}{location}{account}
\newcommand{\RemoteRepoLabel}[3]{%
  \IntensiveCoordinates{Canvas}{c}{#2}
  \node[Blue4, repo label, anchor=#1] at (c)
    {\tt https://github.com/%
      \tikz[baseline={(n.base)},inner sep=.5]{\node[fill=Yellow1](n){#3};}%
      /Pizzas};
}

% [side][bend]{start}{end}
\NewDocumentCommand{\RemoteArrow}{ O{left} O{50} m m }{
  \begin{scope}[transparency group, opacity=.5]
      \draw[-Stealth, line width=20, Brown2] (#3) to[bend #1=#2] (#4);
  \end{scope}
}

% Display the command used.
% [anchor]{location}{text}
\NewDocumentCommand{\Command}{ O{base} m m }{
  \begin{scope}[local to=Canvas]
    \node[scale=\CommandScale, anchor=#1, inner sep=2, fill=Light4]
      (command) at (#2) {\tt \$ #3};
  \end{scope}
}
