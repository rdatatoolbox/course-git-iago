% Draw something looking like a diffed file.

\tikzmath{
  \SignRadius = 1.5;
  \SignBaseHeight = -1.5;
  \FileNameMargins = 2;
  \CodeScale = 1.6;
  \CodeMargins = 2;
  \CodeLineSpacing = 5;
  \SignOffset = 3;
  \DiffFilesSpacing = 8;
}
\colorlet{plus}{Green5}
\colorlet{minus}{Red5}
\colorlet{unchanged}{black}
\colorlet{modified}{Orange3}
\tikzset{
  sign icon/.style={line width=2.0},
  +/.style={plus},
  -/.style={minus},
  0/.style={unchanged},
  m/.style={modified},
}

% Plus icon.
\newsavebox{\PlusBox}
\savebox{\PlusBox}{\tikz[sign icon, every path/.style={+},
                         baseline={(0, \SignBaseHeight)}]{
  \draw (0, -\SignRadius) -- (0, \SignRadius);
  \draw (-\SignRadius, 0) -- (\SignRadius, 0);
}}
\newcommand{\Plus}{\usebox{\PlusBox}}

% Minus icon.
\newsavebox{\MinusBox}
\savebox{\MinusBox}{\tikz[sign icon, every path/.style={-},
                          baseline={(0, \SignBaseHeight)}]{
  \path (0, -\SignRadius) -- (0, \SignRadius);
  \draw (-\SignRadius, 0) -- (\SignRadius, 0);
}}
\newcommand{\Minus}{\usebox{\MinusBox}}

% Tilde icon.
\newsavebox{\TildeBox}
\savebox{\TildeBox}{\tikz[sign icon, baseline={(0, \SignBaseHeight)}]{
  \path (0, -\SignRadius) -- (0, \SignRadius);
  \tikzmath{
    \d = .2*\SignRadius;
    \a = 50;
    \l = 1.5*\SignRadius;
  }
  \coordinate (start) at (-\SignRadius, -\d);
  \coordinate (end) at  (\SignRadius, \d);
  \draw[m] (start)
        .. controls ($(start) + (\a:\l)$) and ($(end) + (180+\a:\l)$)
        .. (end);
}}
\newcommand{\Tilde}{\usebox{\TildeBox}}

% [mode][anchor][name]{location}{name}{mod/lines}
\NewDocumentCommand{\Diff}{ O{0} O{center} O{file} m m m }{
  % Not exactly the corner we expect, but at least the one of first line.

  \node[anchor=#2] (#3) at (#4) {\tikz{

    % Scope the lines to calculate their bounding box.
      \begin{scope}[local bounding box=fileview,
                    every node/.style={scale=\CodeScale}]

        \foreach \sign/\line [count=\i] in {#6} {\ifcsempty{sign}{}{
          % First line has special positionning.
            \ifnumcomp{\i}{=}{1}{
              \node[anchor=north west,\sign] (line-1) {\tt \line};
            }{
              \tikzmath{\prev = int(\i - 1);}
              \node[below=\CodeLineSpacing of line-\prev.base west, \sign,
                anchor=base west]
                  (line-\i) {\tt \line};
            }
        }}

    % File background.
    \begin{pgfonlayer}{background}
    \draw[fill=Light2] ($(fileview.south west) - (\CodeMargins, \CodeMargins)$) rectangle
      ($(fileview.north east) + (\CodeMargins, \CodeMargins)$);
    \end{pgfonlayer}

    \end{scope}

    % Setup file name label.
      \node[above=\FileNameMargins+\CodeMargins of line-1.north west,
      scale=\FileNameScale, anchor=south west, #1]
        (filename) {\tt #5};
    \coordinate (pad) at (\FileNameMargins-\eps, \FileNameMargins);
    \coordinate (lower) at ($(filename.south west) - (pad)$);
    \coordinate (upper) at ($(filename.north east) + (pad)$);
    \draw (lower) -- (lower |- upper) -- (upper) -- (upper |- lower);
    \ifstrequal{#1}{0}{}{%
      \node[anchor=base east] at ($(filename.base west) + (-\SignOffset, .5)$)
      {\ifstrequal{#1}{m}{\Tilde}{\ifstrequal{#1}{+}{\Plus}{\Minus}}};
    }

    % Second pass to append diff signs.
    \foreach \sign/\line [count=\i] in {#6} {\ifcsempty{sign}{}{
      \node[left=\SignOffset of line-\i.base west, anchor=base east]
        {\ifdefstring{\sign}{m}{\Tilde}
        {\ifdefstring{\sign}{+}{\Plus}
        {\ifdefstring{\sign}{-}{\Minus}{}}}};
    }}

  }};


}

