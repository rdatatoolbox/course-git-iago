% Draw something looking like a diffed file.

\tikzmath{
  \SignRadius = 1.5;
  \SignBaseHeight = -1.5;
  \FileNameMargins = 2;
  \CodeScale = 1.4;
  \CodeMargins = 2;
  \SignOffset = 3;
}
\colorlet{plus}{Green5}
\colorlet{minus}{Red5}
\colorlet{unchanged}{Dark4}
\colorlet{modified}{Orange3}
\colorlet{conflict}{Purple3}
\tikzset{
  sign icon/.style={line width=2.0},
  +/.style={plus},
  -/.style={minus},
  0/.style={unchanged},
  m/.style={modified},
  c/.style={conflict},
}

% Plus icon.
\newsavebox{\PlusBox}
\savebox{\PlusBox}{\tikz[sign icon, every path/.style={+},
                         baseline={(0, \SignBaseHeight)}]{
  \draw (0, -\SignRadius) -- (0, \SignRadius);
  \draw (-\SignRadius, 0) -- (\SignRadius, 0);
}}
\newcommand{\Plus}{\usebox{\PlusBox}}

% Minus icon.
\newsavebox{\MinusBox}
\savebox{\MinusBox}{\tikz[sign icon, every path/.style={-},
                          baseline={(0, \SignBaseHeight)}]{
  \path (0, -\SignRadius) -- (0, \SignRadius);
  \draw (-\SignRadius, 0) -- (\SignRadius, 0);
}}
\newcommand{\Minus}{\usebox{\MinusBox}}

% Tilde icon.
\newsavebox{\TildeBox}
\savebox{\TildeBox}{\tikz[sign icon, baseline={(0, \SignBaseHeight)}]{
  \path (0, -\SignRadius) -- (0, \SignRadius);
  \tikzmath{
    \d = .2*\SignRadius;
    \a = 50;
    \l = 1.5*\SignRadius;
  }
  \coordinate (start) at (-\SignRadius, -\d);
  \coordinate (end) at  (\SignRadius, \d);
  \draw[m] (start)
        .. controls ($(start) + (\a:\l)$) and ($(end) + (180+\a:\l)$)
        .. (end);
}}
\newcommand{\Tilde}{\usebox{\TildeBox}}

% Lightning icon.
\newsavebox{\LightningBox}
\savebox{\LightningBox}{\tikz[sign icon, baseline={(0, \SignBaseHeight)}]{
  \tikzmath{
    \a = 80;  \AL = 1.5;
    \b = 75; \BL = .3;
    \c = 30;  \CL = 1;
  }
  \coordinate (A) at (\a:\AL*\SignRadius);
  \coordinate (B) at (\b:\BL*\SignRadius);
  \coordinate (C) at (\c:\CL*\SignRadius);
  \coordinate (A') at (\a-180:\AL*\SignRadius);
  \coordinate (B') at (\b-180:\BL*\SignRadius);
  \coordinate (C') at (\c-180:\CL*\SignRadius);
  \fill[c] (A) -- (B) -- (C) -- (A') -- (B') -- (C') -- cycle;
}}
\newcommand{\Lightning}{\usebox{\LightningBox}}

% Phantom icon, useful for alignment.
\newcommand{\PhantomSign}{\phantom{\usebox{\PlusBox}}}

% [mode][anchor][name][line spacing]{location}{name}{mod/lines}{epilog}
\NewDocumentCommand{\Diff}{ O{0} O{center} O{file} O{5} m m m +m }{
  % Not exactly the corner we expect, but at least the one of first line.

  \AutomaticCoordinates{#3}{#5}
  \node[anchor=#2] (#3) at (#3) {\tikz{

    % Scope the lines to calculate their bounding box.
      \begin{scope}[local bounding box=fileview,
                    every node/.style={scale=\CodeScale}]

        \foreach \mod/\line [count=\i] in {#7} {\ifcsempty{mod}{}{
          % Define this for \dhi.
          \def\DiffBackgroundCol{\ifdefstring{\mod}{+}{fill=plus}
                                {\ifdefstring{\mod}{-}{fill=minus}
                                {\ifdefstring{\mod}{0}{fill=none}
                                {\ifdefstring{\mod}{m}{fill=modified}
                                {}}}}} % For \dhi.
          % First line has special positionning.
          \ifnumcomp{\i}{=}{1}{
            \node[anchor=north west,\mod] (line-1) {\Code{\line}};
          }{
            \tikzmath{\prev = int(\i - 1);}
            \node[below=#4 of line-\prev.base west, \mod,
              anchor=base west]
                (line-\i) {\Code{\line}};
          }
        }}

    % File background.
    \begin{pgfonlayer}{background}
    \draw[fill=Light2] ($(fileview.south west) - (\CodeMargins, \CodeMargins)$) rectangle
      ($(fileview.north east) + (\CodeMargins, \CodeMargins)$);
    \end{pgfonlayer}

    \end{scope}

    % Setup file name label.
      \node[above=\FileNameMargins+\CodeMargins of line-1.north west,
            scale=\FileNameScale, anchor=south west, #1]
        (filename) {\Code{#6}};
    \coordinate (pad) at (\FileNameMargins-\eps, \FileNameMargins);
    \coordinate (lower) at ($(filename.south west) - (pad)$);
    \coordinate (upper) at ($(filename.north east) + (pad)$);
    \draw (lower) -- (lower |- upper) -- (upper) -- (upper |- lower);
    \begin{pgfonlayer}{background}
      \fill[white] ($(lower) + (0, \eps)$) rectangle (upper);
    \end{pgfonlayer}
    \node[anchor=base east] at ($(filename.base west) + (-\SignOffset, .5)$)
      {\ifstrequal{#1}{0}{\PhantomSign}
      {\ifstrequal{#1}{m}{\Tilde}
      {\ifstrequal{#1}{c}{\Lightning}
      {\ifstrequal{#1}{+}{\Plus}{\Minus}}}}};

    % Second pass to append diff signs.
    \foreach \sign/\line [count=\i] in {#7} {\ifcsempty{sign}{}{
      \node[left=\SignOffset of line-\i.base west, anchor=base east]
        {\ifdefstring{\sign}{0}{\PhantomSign}
        {\ifdefstring{\sign}{m}{\Tilde}
        {\ifdefstring{\sign}{c}{\Lightning}
        {\ifdefstring{\sign}{+}{\Plus}
        {\ifdefstring{\sign}{-}{\Minus}{}}}}}};
    }}

  % Epilog commands to be executed within the context of this local picture.
  % TODO: no sub-tikz should be used,
  % but this would need to reposition all diffed files.
  #8

  }};


}

% Highlight changed content in diffed line (uses \CurrentMod)
\newcommand{\dhi}[1]{%
  \tikz[baseline={(n.base)}]{%
    \node[fill=Blue1, fill opacity=0.3, text opacity=1,
          scale=1/\CodeScale] (n) {\vphantom{My}\Code{#1}};}%
}
