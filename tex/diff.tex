% Draw something looking like a diffed file.

\tikzmath{
  \SignRadius = 1.5;
  \SignBaseHeight = -1.5;
  \FileNameMargins = 2;
  \CodeScale = 1.4;
  \CodeMargins = 2;
  \SignOffset = 3;
}
\colorlet{plus}{Green5}
\colorlet{minus}{Red5}
\colorlet{unchanged}{Dark4}
\colorlet{modified}{Orange3}
\colorlet{conflict}{Purple3}
\tikzset{
  sign icon/.style={line width=2.0},
  +/.style={plus},
  -/.style={minus},
  0/.style={unchanged},
  m/.style={modified},
  c/.style={conflict},
}

% Plus icon.
\newsavebox{\PlusBox}
\savebox{\PlusBox}{\tikz[sign icon, every path/.style={+},
                         baseline={(0, \SignBaseHeight)}]{
  \draw (0, -\SignRadius) -- (0, \SignRadius);
  \draw (-\SignRadius, 0) -- (\SignRadius, 0);
}}
\newcommand{\Plus}{\usebox{\PlusBox}}

% Minus icon.
\newsavebox{\MinusBox}
\savebox{\MinusBox}{\tikz[sign icon, every path/.style={-},
                          baseline={(0, \SignBaseHeight)}]{
  \path (0, -\SignRadius) -- (0, \SignRadius);
  \draw (-\SignRadius, 0) -- (\SignRadius, 0);
}}
\newcommand{\Minus}{\usebox{\MinusBox}}

% Tilde icon.
\newsavebox{\TildeBox}
\savebox{\TildeBox}{\tikz[sign icon, baseline={(0, \SignBaseHeight)}]{
  \path (0, -\SignRadius) -- (0, \SignRadius);
  \tikzmath{
    \d = .2*\SignRadius;
    \a = 50;
    \l = 1.5*\SignRadius;
  }
  \coordinate (start) at (-\SignRadius, -\d);
  \coordinate (end) at  (\SignRadius, \d);
  \draw[m] (start)
        .. controls ($(start) + (\a:\l)$) and ($(end) + (180+\a:\l)$)
        .. (end);
}}
\newcommand{\Tilde}{\usebox{\TildeBox}}

% Lightning icon.
\newsavebox{\LightningBox}
\savebox{\LightningBox}{\tikz[sign icon, baseline={(0, \SignBaseHeight)}]{
  \tikzmath{
    \a = 80;  \AL = 1.5;
    \b = 75; \BL = .3;
    \c = 30;  \CL = 1;
  }
  \coordinate (A) at (\a:\AL*\SignRadius);
  \coordinate (B) at (\b:\BL*\SignRadius);
  \coordinate (C) at (\c:\CL*\SignRadius);
  \coordinate (A') at (\a-180:\AL*\SignRadius);
  \coordinate (B') at (\b-180:\BL*\SignRadius);
  \coordinate (C') at (\c-180:\CL*\SignRadius);
  \fill[c] (A) -- (B) -- (C) -- (A') -- (B') -- (C') -- cycle;
}}
\newcommand{\Lightning}{\usebox{\LightningBox}}

% Phantom icon, useful for alignment.
\newcommand{\PhantomSign}{\phantom{\usebox{\PlusBox}}}

% [mode][name][line spacing]{location}{name}{mod/lines}{epilog}
\NewDocumentCommand{\Diff}{ O{0} O{file} O{5} m m m +m }{

  % Not exactly the corner we expect, but at least the one of first line.
  \AutomaticCoordinates{#2}{#4}

  \begin{scope}[every node/.style={scale=\CodeScale}]
      \foreach \mod/\line [count=\i] in {#6} {\ifcsempty{mod}{}{
        % Define this for \dhi.
        \def\DiffBackgroundCol{\ifdefstring{\mod}{+}{fill=plus}
                              {\ifdefstring{\mod}{-}{fill=minus}
                              {\ifdefstring{\mod}{0}{fill=none}
                              {\ifdefstring{\mod}{m}{fill=modified}
                              {}}}}} % For \dhi.
        \ifnumcomp{\i}{=}{1}{
          % First line has special positionning.
          \node[anchor=base west,\mod,
                alias=#2-line-1,
                alias=last-line,
                alias=#2-last-line,
                alias=widest-line,
                alias=#2-widest-line]
            (line-1) at (#2) {\Code{\line}};
        }{
          \tikzmath{\prev = int(\i - 1);}
          \node[below=#3 of last-line.base west, \mod, anchor=base west,
                alias=#2-line-\i,
                alias=last-line,
                alias=#2-last-line]
                (line-\i) {\Code{\line}};
          % Keep track which line is the widest.
          \tikzmath{
            coordinate \w, \e, \W, \E;
            \w = (last-line.base west);
            \e = (last-line.base east);
            \W = (widest-line.base west);
            \E = (widest-line.base east);
            \d = \ex - \wx;
            \D = \Ex - \Wx;
            \greater = \D < \d;
          }
          \ifdefstring{\greater}{1}{
            \node also [alias=widest-line, alias=#2-widest-line] (last-line);
          }
        }
      }}
  \end{scope}

  % File background.
  \begin{pgfonlayer}{background}
    \coordinate (pad) at (\CodeMargins, \CodeMargins);
    \coordinate (lo) at ($(last-line.south west) - (pad)$);
    \coordinate (up) at ($(widest-line.east |- line-1.north) + (pad)$);
    \node[draw, fill=Light2, fit=(lo)(up), alias=#2-content] (content) {};
  \end{pgfonlayer}

  % Setup file name label.
  \node[above=\FileNameMargins+\CodeMargins of line-1.north west,
        scale=\FileNameScale, anchor=south west, alias=#2-filename-label, #1]
        (filename-label) {\Code{#5}};
  \coordinate (pad) at (\FileNameMargins, \FileNameMargins);
  \coordinate (lower) at ($(filename-label.south west) - (pad)$);
  \coordinate (upper) at ($(filename-label.north east) + (pad)$);
  \node[draw, fit={(lower) (upper)}, alias=#2-filename] (filename) {};
  \begin{pgfonlayer}{background}
    \fill[white] ($(lower) + (0, \eps)$) rectangle (upper);
  \end{pgfonlayer}
  \node[anchor=base east] at ($(filename-label.base west) + (-\SignOffset, .5)$)
    {\ifstrequal{#1}{0}{\PhantomSign}
    {\ifstrequal{#1}{m}{\Tilde}
    {\ifstrequal{#1}{c}{\Lightning}
    {\ifstrequal{#1}{+}{\Plus}{\Minus}}}}};

  % Second pass to append diff signs.
  \foreach \sign/\line [count=\i] in {#6} {\ifcsempty{sign}{}{
    \node[left=\SignOffset of line-\i.base west, anchor=base east]
      (sign-\i)
      {\ifdefstring{\sign}{0}{\PhantomSign}
      {\ifdefstring{\sign}{m}{\Tilde}
      {\ifdefstring{\sign}{c}{\Lightning}
      {\ifdefstring{\sign}{+}{\Plus}
      {\ifdefstring{\sign}{-}{\Minus}{}}}}}};
  }}

  % Construct one total bounding box node.
  \coordinate (lo) at (sign-1.west |- content.south);
  \coordinate (up) at (content.east |- filename.north);
  \node[fit=(lo)(up)] (#2) {};

  % Epilog
  #7

}

% Highlight changed content in diffed line (uses \CurrentMod)
\newcommand{\dhi}[1]{%
  \tikz[baseline={(n.base)}]{%
    \node[fill=Blue1, fill opacity=0.3, text opacity=1,
          scale=1/\CodeScale] (n) {\vphantom{My}\Code{#1}};}%
}
