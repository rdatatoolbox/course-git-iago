% Draw something looking like a files hierarchy.
% Not everything is possible yet, but it's maybe useful at least.
\tikzmath{
  \FileNameScale = 1.9;
  \FileSpacing = 15;
  \IntoFileTreeSpacing = 15;
  \IconSize = 14;
}
\tikzset{
  file icon/.style 2 args={
    line width=1.5,
    rounded corners,
    line cap=round,
    every path/.style={fill=#1, draw=#2},
    },
  file icon +/.style={file icon={#1}{plus}},
  file icon -/.style={file icon={#1}{minus}},
  file icon 0/.style={file icon={#1}{unchanged}},
  file icon m/.style={file icon={#1}{modified}},
  file tree line/.style={line width=1.5, line cap=round},
}

% Pad icons to given width.
\newcommand{\FileIconFrame}{\path[fill=none, draw=none] (0, 0)
  rectangle (\IconSize, 0);}

% Folder icon.
\NewDocumentCommand{\Folder}{ O{0} }{%
\tikz[file icon #1={Orange2}, baseline={(0,.35*\IconSize)}]{
  \FileIconFrame
  \path (0, 0) rectangle (0.4*\IconSize, .75*\IconSize);
  \path (0, 0) rectangle (0.85*\IconSize, .65*\IconSize);
  \begin{scope}[cm={1, 0, .25, .85, (0, 0)}] % Transformation matrix.
    \path (0, 0) rectangle (0.85*\IconSize, .65*\IconSize);
  \end{scope}
}}

% File icon.
\NewDocumentCommand{\File}{ O{0} }{%
\tikz[file icon #1={Light2}, baseline={(0,.45*\IconSize)}]{{
  \FileIconFrame
  \tikzmath{
    \W = .7*\IconSize;
    \H = .9*\IconSize;
    \o = .15*\IconSize; % horizontal offset
    \n = 4;   % number of lines
    \s = 2; % lines shortening
    \h = \H / (\n + 1); % lines spacing
  }
  \path (\o, 0) rectangle (\W+\o, \H);
  \foreach \i in {1,...,\n} {%
    \path (\o+\s, \i*\h) -- (\W+\o-\s, \i*\h);
  }
}}}

% Insert file/folder icon with a name and correct modification status.
% Type contains keywords like 'file|folder' and 'last' checked with \IfSubStr.
% 'mod' is either '0+-m' TODO: c for 'conflict'.
% [type][mode]{location}{name}{filename}
\NewDocumentCommand{\FileLine}{ O{file} O{0} m m m }{{

  % Icon.
  \AutomaticCoordinates{#4}{#3}
  \node[anchor=north west,
        alias=\LatestFileTree-#4-icon] (#4-icon) at (#4)
    {\IfSubStr{#1}{file}{\File[#2]}{\Folder[#2]}};

  % Filename.
  \IfSubStr{#1}{file}{\coordinate (offset) at (3, 5);}
                     {\coordinate (offset) at (3, 3);}
  \node[anchor=base west, scale=\FileNameScale, #2,
        alias=\LatestFileTree-#4-filename] (#4-filename)
    at ($(#4-icon.south east) + (offset)$)
    {\tt#5\IfSubStr{#1}{folder}{/}{}};

  % Status sign.
  \ifdefstring{#2}{0}{}{
    \node[anchor=base west, right=\SignOffset of #4-filename]
      {\ifdefstring{#2}{m}{\Tilde}
      {\ifdefstring{#2}{+}{\Plus}
      {\ifdefstring{#2}{-}{\Minus}
      {[#2]}}}};} % (fallback to writing unexpected mode literally)

}}

% Chain files together, like commits in the repo.
% [anchor][name]{location}{type/mod/name/filename list}
\NewDocumentCommand{\FileTree}{ O{north west} O{files} m m }{

  % Update this global variable so that files icons, names, signs etc.
  % can be aliased with (filetreename-nodename) in addition to just (nodename).
  \def\LatestFileTree{#2}

  \AutomaticCoordinates{filetreeloc}{#3}
  \node[anchor=#1] (#2) at (filetreeloc) {\tikz[remember picture]{%
    \foreach \type/\mod/\name/\filename [count=\i, remember=\name as \lastname]
             in {#4}{\ifcsempty{filename}{}{%
      \ifnumcomp{\i}{=}{1}{
        % Keep track of current icon position.
        \coordinate (current) at (0, 0);
        \FileLine[\type][\mod]{current}{\name}{\filename}
      }{
        \coordinate[below=\FileSpacing of current] (current);
        % type with:
        %   'stepin' to enter into last folder.
        %   'connect' to append sibling into currently 'stepped in' folder.
        %   'last' to mark last sibling within a 'steeped in' folder.
        % TODO: 'step out' and continuing outer line within sub-subfolders.
        \IfSubStr{\type}{stepin}{
          \coordinate[right=\IntoFileTreeSpacing of current] (current);
        }{}
        \FileLine[\type][\mod]{current}{\name}{\filename}
        \IfSubStr{\type}{stepin}{%
          % First parenting line within the folder.
          \coordinate[below=2 of \lastname-icon.south] (s);
        }{\IfSubStr{\type}{connect}{%
          % Assumes an 'anchor' already exists.
          \coordinate (s) at (anchor);
        }{}}
        \ifboolexpr{ test {\IfSubStr{\type}{stepin}}
                  or test {\IfSubStr{\type}{connect}} }{
          % Draw the actual line, creating 'anchor' for next time.
          \coordinate[left=3 of \name-icon.west] (e);
          \coordinate (anchor) at (s|-e);
          % Only color whole line for last items in the chain.
          \IfSubStr{\type}{last}{
            \draw[file tree line, \mod] (s) -- (anchor) -- (e);
          }{
            \draw[file tree line] (s) -- (anchor);
            \draw[file tree line, \mod] (anchor) -- (e);
          }
        }{}
      }
    }}
  }};

}
