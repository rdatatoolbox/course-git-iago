% Global frame structure.
% Invoke on every change to redraw.

\tikzmath{
  \ScreenWidth = 400;
  \ScreenHeight = 300;
  \TitleBarHeight = 35;
  \TitleScale = 5.0;
  \SubTitleScale = 3.5;
  \PageNumScale = 2.0;
  \ProgressHeight = 1.7;
  \CanvasMargin = 5;
  \LargeScale = 3.0;
  \NormalScale = 2.0;
}
\tikzset{
  progress made/.style={fill=Orange2},
  progress remaining/.style={fill=Yellow1},
}

% Uses \TitleText, \SubTitleText, \PageNumText.
% type=bare for non-regular steps with only a blank 'Screen'.
% type=title for transitions between slides.
% [type]{progress}{content}
\NewDocumentCommand{\Step}{ O{} m +m }{

\begin{step}%
\begin{tikzpicture}

% Whole screen, setting the page size.
\node[anchor=south west] (Screen) {\tikz{
  \path (0, 0) rectangle (\ScreenWidth, \ScreenHeight)}};

\IfSubStr{#1}{bare}{}{\IfSubStr{#1}{title}{

  \IntensiveCoordinates{Screen}{c}{0, .15}
  \node[scale=7, Dark3] at (c) {\TitleText};

  \IntensiveCoordinates{Screen}{lower}{-.7, -.005}
  \IntensiveCoordinates{Screen}{upper}{+.7, +.005}
  \path[progress remaining] (lower) rectangle (upper);
  \coordinate (mid) at ($(lower)!#2!(upper)$);
  \path[progress made] (lower) rectangle (mid|-upper);

}{


    % Tikz's 'z' system is not awesome, so many layers end up very specific :\
    \pgfdeclarelayer{highlight-behind}
    \pgfdeclarelayer{background}
    \pgfdeclarelayer{command-background}
    \pgfdeclarelayer{command-text}
    \pgfsetlayers{
      highlight-behind,
      background,
      main,
      command-background,
      command-text
    }

    % Title bar.
    \node[anchor=north] (TitleBar) at (Screen.north) {\tikz{
      \fill[Dark3] (0, 0) rectangle (\ScreenWidth, \TitleBarHeight);}};

    % Title.
    \node[Light2, anchor=base west, scale=\TitleScale,
          right=10 of TitleBar.west] (Title) {\bf \TitleText};

    % SubTitle.
    \node[Light2, anchor=base east, scale=\SubTitleScale,
          left=10 of TitleBar.east] (SubTitle) {\SubTitleText};

    % Page number.
    \node[Dark4, inner sep=5, anchor=south east, scale=\PageNumScale]
      (PageNum) at (Screen.south east) {\bf \PageNumText};

    % Progress bar,
    % fix vertical borders white pixel lines with epsilon shifts.
    \coordinate (upper) at
      ($(TitleBar.south east) + (-\eps, \ProgressHeight)$);
    \coordinate[right=\eps of TitleBar.south west] (lower);
    \path[progress remaining] (lower) rectangle (upper);
    \coordinate[left=(1-#2)*\ScreenWidth of upper] (upper);
    \path[progress made] (lower) rectangle (upper);

    % The "Canvas" refers to only the white area reserved for actual drawing,
    % minus a short margin.
    \coordinate (offset) at (\CanvasMargin, \CanvasMargin);
    \node[anchor=south west] at (offset) (Canvas) {\tikz{
      \path (offset) rectangle ($(TitleBar.south east) - (offset)$);}};

}}

#3

\end{tikzpicture}%
\end{step}

}

% Factorize bounding box highlighting procedures.
% [padding][opacity]{lower}{upper}
\NewDocumentCommand{\HighlightSquare}{ O{5} O{1} m m }{
  \begin{pgfonlayer}{highlight-behind}%
    \ifstrequal{#2}{1}{}{\begin{scope}[opacity=#2]}
    \coordinate (pad) at (#1, #1);
    \coordinate (lower) at ($(#3) - (pad)$);
    \coordinate (upper) at ($(#4) + (pad)$);
    \draw[Light5, line width=1, fill=Yellow1] (lower) rectangle (upper);
    \ifstrequal{#2}{1}{}{\end{scope}}
  \end{pgfonlayer}
}

% Hollow bounding box, useful for branches labels.
% [padding][opacity][color]{node}
\NewDocumentCommand{\HighlightSquareRing}{ O{1} O{1} O{Yellow1} m }{
    \coordinate (pad) at (#1, #1);
    \coordinate (lower) at ($(#4.south west) - (pad)$);
    \coordinate (upper) at ($(#4.north east) + (pad)$);
    \path[draw=#3, draw opacity=#2, line width=4] (lower) rectangle (upper);
}

% Highlight with background shade instead.
% [padding]{node}
\NewDocumentCommand{\HighlightShade}{ O{5} m }{
  \begin{pgfonlayer}{highlight-behind}%
    \coordinate (pad) at (#1, #1);
    \coordinate (lower) at ($(#2.south west) - (pad)$);
    \coordinate (upper) at ($(#2.north east) + (pad)$);
    \shade[inner color=Yellow1] (lower) rectangle (upper);
  \end{pgfonlayer}
}

% Automatic coordinates are interpreted as intensive to Canvas,
% unless they contain no comma (node name) or are a calculation with '$' or '|'.
% If they contain '=', they are interpreted as relative specifications
% like 'below=5 of node'
% {name}{location}
\newcommand{\AutomaticCoordinates}[2]{
  \IfSubStr{#2}{=}{
    \coordinate[#2] (#1);
  }{\ifboolexpr{test {\IfSubStr{#2}{$}} % Breaks syntax coloring -_-"
             or test {\IfSubStr{#2}{|}}
         or not test {\IfSubStr{#2}{,}}}
    {\coordinate (#1) at (#2);}
    {\IntensiveCoordinates{Canvas}{#1}{#2}}
  }
}

