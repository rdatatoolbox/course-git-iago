% Global frame structure.
% Invoke on every change to redraw.

\tikzmath{
  \ScreenWidth = 400;
  \ScreenHeight = 300;
  \TitleBarHeight = 35;
  \TitleScale = 5.0;
  \SubTitleScale = 3.5;
  \PageNumScale = 2.0;
  \ProgressHeight = 1.7;
  \CanvasMargin = 5;
  \LargeScale = 3.0;
  \NormalScale = 2.0;
}
\tikzset{
  progress made/.style={fill=Orange2},
  progress remaining/.style={fill=Yellow1},
}

% Uses \TitleText, \SubTitleText, \PageNumText, \Progress
% {content}
\NewDocumentCommand{\Step}{ +m }{

\begin{step}%
\begin{tikzpicture}

  \pgfdeclarelayer{highlight}
  \pgfdeclarelayer{background}
  \pgfsetlayers{highlight, background, main}

  % Set off global structure first.

  % Whole screen, setting the page size.
  \node[anchor=south west] (Screen) {\tikz{
    \path (0, 0) rectangle (\ScreenWidth, \ScreenHeight)}};

  % Title bar.
  \node[anchor=north] (TitleBar) at (Screen.north) {\tikz{
    \fill[Dark3] (0, 0) rectangle (\ScreenWidth, \TitleBarHeight);}};

  % Title.
  \node[Light2, anchor=base west, scale=\TitleScale,
        right=10 of TitleBar.west] (Title) {\bf \TitleText};

  % SubTitle.
  \node[Light2, anchor=base east, scale=\SubTitleScale,
        left=10 of TitleBar.east] (SubTitle) {\bf \SubTitleText};

  % Page number.
  \node[Dark4, inner sep=5, anchor=south east, scale=\PageNumScale]
    (PageNum) at (Screen.south east) {\bf \PageNumText};

  % Progress bar,
  % fix vertical borders white pixel lines with epsilon shifts.
  \coordinate (upper) at
    ($(TitleBar.south east) + (-\eps, \ProgressHeight)$);
  \coordinate[right=\eps of TitleBar.south west] (lower);
  \path[progress remaining] (lower) rectangle (upper);
  \coordinate[left=(1-\Progress)*\ScreenWidth of upper] (upper);
  \path[progress made] (lower) rectangle (upper);

  % The "Canvas" refers to only the white area reserved for actual drawing,
  % minus a short margin.
  \coordinate (offset) at (\CanvasMargin, \CanvasMargin);
  \node[anchor=south west] at (offset) (Canvas) {\tikz{
    \path (offset) rectangle ($(TitleBar.south east) - (offset)$);}};

  #1

\end{tikzpicture}%
\end{step}

}

% Factorize bounding box highlighting procedures.
% [padding]{lower}{upper}
\NewDocumentCommand{\HighlightSquare}{ O{5} m m }{
  \begin{pgfonlayer}{highlight}%
    \coordinate (offset) at (#1, #1);
    \coordinate (lower) at ($(#2) - (offset)$);
    \coordinate (upper) at ($(#3) + (offset)$);
    \draw[Light5, line width=1, fill=Yellow1] (lower) rectangle (upper);
  \end{pgfonlayer}
}
\NewDocumentCommand{\HighlightShade}{ O{5} m }{
  \begin{pgfonlayer}{highlight}%
    \coordinate (offset) at (#1, #1);
    \coordinate (lower) at ($(#2.south west) - (offset)$);
    \coordinate (upper) at ($(#2.north east) + (offset)$);
    \shade[inner color=Yellow1] (lower) rectangle (upper);
  \end{pgfonlayer}
}
